\PassOptionsToPackage{unicode}{hyperref}
\PassOptionsToPackage{hyphens}{url}
%
\documentclass[
  12pt,
  answers  
]{exam}
\usepackage[fleqn]{amsmath}
\usepackage[]{plex-otf}
\usepackage{iftex}
\usepackage[a4paper, margin=2cm]{geometry}
\usepackage{graphicx}
\usepackage[sorting=none, backend=biber]{biblatex}
% \ifPDFTeX
%   \usepackage[T1]{fontenc}
%   \usepackage[utf8]{inputenc}
%   \usepackage{textcomp} % provide euro and other symbols
% \else % if luatex or xetex
%   \usepackage{unicode-math}
%   \defaultfontfeatures{Scale=MatchLowercase}
%   \defaultfontfeatures[\rmfamily]{Ligatures=TeX,Scale=1}
% \fi
% Use upquote if available, for straight quotes in verbatim environments
\IfFileExists{upquote.sty}{\usepackage{upquote}}{}
\IfFileExists{microtype.sty}{% use microtype if available
  \usepackage[]{microtype}
  \UseMicrotypeSet[protrusion]{basicmath} % disable protrusion for tt fonts
}{}
\makeatletter
\makeatother
\usepackage{xcolor}
\IfFileExists{xurl.sty}{\usepackage{xurl}}{} % add URL line breaks if available
\IfFileExists{bookmark.sty}{\usepackage{bookmark}}{\usepackage{hyperref}}
\urlstyle{same} % disable monospaced font for URLs
\setlength{\emergencystretch}{3em} % prevent overfull lines
\providecommand{\tightlist}{%
  \setlength{\itemsep}{0pt}\setlength{\parskip}{0pt}}
\setcounter{secnumdepth}{-\maxdimen} % remove section numbering
\newlength{\cslhangindent}
\setlength{\cslhangindent}{1.5em}
\newlength{\csllabelwidth}
\setlength{\csllabelwidth}{3em}
\newlength{\cslentryspacingunit} % times entry-spacing
\setlength{\cslentryspacingunit}{\parskip}
\newenvironment{CSLReferences}[2] % #1 hanging-ident, #2 entry spacing
 {% don't indent paragraphs
  \setlength{\parindent}{0pt}
  % turn on hanging indent if param 1 is 1
  \ifodd #1
  \let\oldpar\par
  \def\par{\hangindent=\cslhangindent\oldpar}
  \fi
  % set entry spacing
  \setlength{\parskip}{#2\cslentryspacingunit}
 }%
 {}
\usepackage{calc}
\newcommand{\CSLBlock}[1]{#1\hfill\break}
\newcommand{\CSLLeftMargin}[1]{\parbox[t]{\csllabelwidth}{#1}}
\newcommand{\CSLRightInline}[1]{\parbox[t]{\linewidth - \csllabelwidth}{#1}\break}
\newcommand{\CSLIndent}[1]{\hspace{\cslhangindent}#1}
\ifLuaTeX
  \usepackage{selnolig}  % disable illegal ligatures
\fi
\renewcommand{\solutiontitle}{\noindent\textbf{Jawab:}\par\noindent}
\newcommand{\mytitle}{Pancasila (UM 162)}
\newcommand{\theauthor}{Rivo Juicer Wowor}
\newcommand{\affiliation}{00000059635}

% \title{\textbf{\mytitle}}
% \author{\theauthor \
		% \small{\affiliation}}
% \date{}

% \usepackage{fancyhdr}
\lhead{\footnotesize{\textbf{\mytitle}}}%
\rfoot{\thepage}%
\cfoot{}
\lfoot{\footnotesize{\theauthor \hspace{1pt} (\affiliation)}}
\pagestyle{headandfoot}

% \pagestyle{plain}
\setlength{\parindent}{2em}
\renewcommand{\baselinestretch}{1.5}
\hbadness=99999
\addbibresource{../../ref/library.bib}

\begin{document}
	\begin{titlepage}
		\centering
		\vspace{2cm}
		\includegraphics[width=0.5\textwidth]{../../ref/logoUMN.png}\par\vspace{1cm}
		\vspace{1.5cm}
		\large{Midterm Assignment} \par
		\vspace{1cm}
		\LARGE{\textbf{\mytitle}} \par
		\vspace{1.5cm}
		\large{\theauthor} \par
		\small{\emph{\affiliation}} \par
		\vfill
		\today
		\end{titlepage}	

    \begin{questions}
      \question 
      Perhatikan dua definisi berikut ini:
      \begin{enumerate}
        \item Bangsa ialah sejumlah orang banyak yang disatukan oleh kesamaan budaya, sejarah, keturunan, bahasa serta menghuni satu wilayah.
        \item Negara adalah sebuah bangsa atau wilayah yang merupakan satu kesatuan komunitas politis di bawah satu pemerintahan.
      \end{enumerate}

      Berdasarkan kedua definisi tersebut, buat satu argumentasi yang terdiri dari \textbf{tiga paragraf} yang menjelaskan sekaligus mengklaim bahwa:
      \begin{parts}
        \part Pancasila adalah ideologi \emph{bangsa}, atau
        \part Pancasila adalah ideologi \emph{negara}.
      \end{parts}

      Anda hanya boleh memilih salah satu dari (a) atau (b), dan tidak boleh keduanya meski terbuka kemungkinan untuk menunjukkan kelebihan serta kekurangan masing-masing proposisi.

      \begin{solution}
        Untuk bisa membandingkan Pancasila terhadap dua definisi tersebut, kita dapat melihat dalam sudut pandang definisi sebuah ideologi terlebih dahulu. Menurut KBBI, ideologi adalah \emph{kumpulan konsep bersistem yang dijadikan asas pendapat (kejadian) yang memberikan arah dan tujuan untuk kelangsungan hidup} \cite{badanpusatpengembangandanpembinaanbahasaIdeologi}. Oleh karena itu Pancasila bisa disebutkan sebagai sebuah ideologi karena berisi tujuan dan nilai-nilai kehidupan dan bernegara bagi masyarakat Indonesia. Dan juga slogan dari Pancasila \emph{Bhinneka Tunggal Ika} yang berarti "Berbeda beda tetapi tetap satu" menguatkan argumen bahwa Pancasila merupakan sebuah ideologi \emph{bangsa} berdasarkan kedua definisi diatas.

        Tapi jika kita melihat dari sudut pandang sejarah, Pancasila juga bisa dikatakan sebagai ideologi \emph{negara}. Mengapa? Karena Pancasila dibangun oleh para Bapak Pendiri Bangsa serta lembaga-lembaga persiapan kemerdekaan Indonesia sebelum merdeka yang notabene merupakan sebuah komunitas politik. Pada pidato "Lahirnya Panca Sila", Soekarno juga mengembangkan rumusan Pancasila-nya dalam sudut pandang suatu negara ketimbang sudut pandang masyarakat yang beragam. \cite{soekarnoLahirnyaPancaSila}

        Menurut saya sendiri, saya lebih setuju dengan argumen Pancasila adalah sebuah ideologi \emph{bangsa}. Karena jika kita melihat dalam tujuan dan nilai yang terkandung pada sila-sila Pancasila, semuanya memiliki satu kesamaan yaitu menyatukan orang-orang yang berbeda latar belakang; baik itu suku, budaya, ras, dan agama. Sehingga rakyat Indonesia bisa memiliki kesamaan dalam Pancasila yang dapat menjadikan mereka sebagai sebuah Bangsa. Oleh karena itu, \textbf{saya setuju akan pernyataan bahwa Pancasila adalah Ideologi Bangsa}.
      \end{solution}

      \question
      Kitab Undang-Undang Hukum Pidana di Indonesia memberlakukan hukuman mati terhadap tindakan kriminal tertentu, salah satunya adalah korupsi. Pertanyaannya, apakah Anda setuju/tidak dengan pemberlakuan hukuman mati itu? Berikan argumentasi dengan menggunakan sudut pandang etika!
      \begin{solution}
        Hukum diciptakan untuk menjaga ketentraman dan kedamaian antar hidup masyarakat. Dan terdapat juga sanksi bagi orang yang melanggar hukum tersebut. Penerapan sanksi tersebut wajib mengatur seseorang sebagai subyek hukum, dan harus memiliki rasa peri kemanusiaan dalam menghargai harga dan martabat hidup seseorang \cite{siburianHUKUMANMATIDI2021}. Dan hukuman mati banyak menuai kontroversi dan kritik. Ada yang berargumen bahwa hukuman mati harus tetap dilakukan untuk memberikan efek jera bagi para pelaku kejahatan berat, tapi ada juga yang berargumen bahwa hukuman mati harus ditiadakan karena melanggar hak hidup manusia. 

        Jika kita memandang dalam sudut pandang etika, hukuman mati sangatlah bertentangan dengan hak asasi manusia. PBB sendiri membuat sebuah pernyataan yang dituliskan pada Pasal 3 \emph{Universal Declarataion of Human Rights}: \emph{"Everyone has the right to life, liberty and security of person."} (Setiap orang memiliki hak untuk hidup, bebas, dan aman) \cite{nationsUniversalDeclarationHuman}. Hal ini juga dipertegas oleh Sekretaris PBB, António Guterres pada tahun 2017 mengatakan bahwa "Hukuman mati tidak memiliki tempat di abad ke-21" \cite{DeathPenaltyHas2017}. Selain itu, hukuman mati juga tidak selamanya memberikan rasa keadilan yang sesungguhnya bagi korban dan dinilai bertolak belakang dengan nilai moral dan kemanusiaan yang ada. 

        Dari pernyataan diatas, kita bisa dapatkan bahwa dalam perspektif etika, hukuman mati harus ditiadakan kembali. Hukuman mati bukanlah satu-satunya cara untuk menyelesaikan suatu masalah, dan masih banyak alternatif lain yang bisa dilakukan untuk memberikan efek jera bagi pelaku kejahatan berat selain hukuman mati. 
      \end{solution}

      \question
      Era globalisasi dewasa ini membawa perubahan yang sangat besar termasuk dalam hal identitas nasional. Menurut Anda, apakah di era ini ada krisis baik laten maupun yang nyata sudah atau sedang "menghantui" Pancasila sebagai identitas bangsa? Jelaskan dengan singkat!
      \begin{solution}
        Pancasila merupakan dasar negara Indonesia yang lahir berdasarkan nilai-nilai budaya yang sudah ada sejak jaman nenek moyang kita \cite{pujiasmaroiniMENJAGAEKSISTENSIPANCASILA2017}. Dan juga jika kita melihat rumusan yang disampaikan oleh Soekarno, Pancasila merupakan sebuah wujud globalisasi, karena menempatkan posisi Indonesia dalam konteks Internasionalisme \cite[p.~74]{boloPancasilaDalamPendidikan2020}. Tapi bukan berarti Pancasila tidak dihadapi dengan masalah-masalah yang membuatnya terpengaruh sebagai sebuah identitas nasional. Banyak masalah terjadi di era globalisasi ini yang membuat posisi Pancasila sebagai identitas bangsa tergoyahkan.
        
        Yang pertama ialah semangat politik identitas yang terus bermunculan di kalangan masyarakat Indonesia. Politik identitas ini bertujuan untuk menonjolkan suatu kelompok tertentu dan tentunya merusak nilai persatuan dan \emph{kebhinekaan} yang ada pada nilai dan sila Pancasila \cite{situruPancasilaDanTantangan2019}. Yang kedua adalah munculnya sebuah gerakan individualisme ditengah masyarakat Indonesia yang dipengaruhi oleh gaya hidup orang barat. Gerakan ini juga dibarengi dengan munculnya ideologi Kapitalisme-liberalisme yang ada di negara-negara barat dan tentunya membuat banyak orang yang merasa tidak relevan lagi dengan ideologi Pancasila \cite{rosyidinpenguatan}. Yang terakhir adalah munculnya berbagai ideologi radikalisme yang diakibatkan oleh mudahnya mengakses informasi pada era globalisasi ini. Beberapa faktor lain juga mempengaruhi munculnya paham ideologi radikalisme ini seperti faktor pemikiran yang sempit, faktor permasalahan ekonomi, dan juga faktor politik yang berkaitan dengan politik identitas \cite{mediaFaktorPenyebabMunculnya2021}. Sehingga, orang yang sudah terpengaruh oleh paham radikalisme ini merasa Pancasila tidak sesuai dengan kaidah hukum yang dipahami oleh ideologi radikalnya dan menyebabkan tergesernya posisi Pancasila sebagai identitas nasional.


      \end{solution}

      \question
      Selama ini kebhinekaan bangsa belum sepenuhnya dipahami oleh segenap warga Indonesia. Padahal kemerdekaan republik Indonesia diraih bukan berkat perjuangan satu kelompok, melainkan banyak pihak dengan beragam latar belakang-etnis, agama, kelas sosial hingga afiliasi politik. \emph{Jelaskan kontribusi terpenting yang diwariskan para tokoh minoritas anggota BPUPK dan PPKI untuk kemerdekaan Republik Indonesia dan berikan alasan yang rasional dan argumentatif kontribusi mereka bagi nilai-nilai kebhinekaan Pancasila di jaman now ini!}     
      \begin{solution}
        Dalam sejarah pendirian negara Indonesia, terdapat dua lembaga yang memiliki peran sangat penting. Yang pertama adalah BPUPK (Badan Penyelidik Usaha-Usaha Persiapan Kemerdekaan) yang didirikan pada tanggal 29 April 1945 dan PPKI (Panitia Persiapan Kemerdekaan Indonesia) yang didirikan pada tanggal 7 Agustus 1945 untuk mengganti BPUPKI. Kedua lembaga ini terdiri dari orang-orang yang memiliki bermacam-macam latar belakang baik suku, agama, etnis dan budaya.

        Salah satu anggotanya adalah Liem Koen Hian, seorang wartawan \emph{Soeara Publiek} dan juga pendiri Partai Tionghoa Indonesia (PTI). Ia adalah salah satu dari empat orang perwakilan peranakan Tionghoa pada panitia BPUPK. Dalam sidang kedua BPUPK tanggal 11 Juli 1945, Liem Koen Hian berpidato mengenai posisi orang-orang dengan etnis Tionghoa dalam masyarakat Indonesia. Ia berpendapat bahwa semua kaum Tionghoa yang lahir di Indonesia berhak memiliki kewarganegaraan Indonesia tanpa terlebih dahulu meminta pendapat masing-masing orang Tionghoa \cite{dawaOrangTionghoaDalam2009}. 

        Berdasarkan salah satu kontribusi dari anggota minoritas BPUPKI tersebut, kita dapat menyimpulkan bahwa orang Indonesia tidak hanyalah orang-orang pribumi dan etnis asli Indonesia saja. Tapi juga orang-orang peranakan lahir dan besar di Indonesia yang memiliki jiwa nasionalisme bagi negara ini. Sila ketiga dan juga semboyan \emph{Bhinneka Tunggal Ika} dari Pancasila sendiri mencerminkan hal tersebut. Meskipun orang Indonesia memiliki latar belakang etnis, suku, budaya, agama, dan ras yang sangat beragam, tapi kita tetaplah satu sebagai masyarakat Bangsa Indonesia.
      \end{solution}
    \end{questions}

    \pagebreak
\printbibliography

		
\end{document}

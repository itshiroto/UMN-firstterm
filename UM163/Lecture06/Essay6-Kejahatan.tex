% Options for packages loaded elsewhere
\PassOptionsToPackage{unicode}{hyperref}
\PassOptionsToPackage{hyphens}{url}
%
\documentclass[
  12pt  
]{article}
\usepackage[fleqn]{amsmath}
\usepackage[]{plex-otf}
\usepackage{iftex}
\usepackage[a4paper, margin=2.5cm]{geometry}

% \ifPDFTeX
%   \usepackage[T1]{fontenc}
%   \usepackage[utf8]{inputenc}
%   \usepackage{textcomp} % provide euro and other symbols
% \else % if luatex or xetex
%   \usepackage{unicode-math}
%   \defaultfontfeatures{Scale=MatchLowercase}
%   \defaultfontfeatures[\rmfamily]{Ligatures=TeX,Scale=1}
% \fi
% Use upquote if available, for straight quotes in verbatim environments
\IfFileExists{upquote.sty}{\usepackage{upquote}}{}
\IfFileExists{microtype.sty}{% use microtype if available
  \usepackage[]{microtype}
  \UseMicrotypeSet[protrusion]{basicmath} % disable protrusion for tt fonts
}{}
\makeatletter
\makeatother
\usepackage{xcolor}
\IfFileExists{xurl.sty}{\usepackage{xurl}}{} % add URL line breaks if available
\IfFileExists{bookmark.sty}{\usepackage{bookmark}}{\usepackage{hyperref}}
\urlstyle{same} % disable monospaced font for URLs
\setlength{\emergencystretch}{3em} % prevent overfull lines
\providecommand{\tightlist}{%
  \setlength{\itemsep}{0pt}\setlength{\parskip}{0pt}}
\setcounter{secnumdepth}{-\maxdimen} % remove section numbering
\newlength{\cslhangindent}
\setlength{\cslhangindent}{1.5em}
\newlength{\csllabelwidth}
\setlength{\csllabelwidth}{3em}
\newlength{\cslentryspacingunit} % times entry-spacing
\setlength{\cslentryspacingunit}{\parskip}
\newenvironment{CSLReferences}[2] % #1 hanging-ident, #2 entry spacing
 {% don't indent paragraphs
  \setlength{\parindent}{0pt}
  % turn on hanging indent if param 1 is 1
  \ifodd #1
  \let\oldpar\par
  \def\par{\hangindent=\cslhangindent\oldpar}
  \fi
  % set entry spacing
  \setlength{\parskip}{#2\cslentryspacingunit}
 }%
 {}
\usepackage{calc}
\newcommand{\CSLBlock}[1]{#1\hfill\break}
\newcommand{\CSLLeftMargin}[1]{\parbox[t]{\csllabelwidth}{#1}}
\newcommand{\CSLRightInline}[1]{\parbox[t]{\linewidth - \csllabelwidth}{#1}\break}
\newcommand{\CSLIndent}[1]{\hspace{\cslhangindent}#1}
\ifLuaTeX
  \usepackage{selnolig}  % disable illegal ligatures
\fi

\newcommand{\mytitle}{Pendapat mengenai Kejahatan dengan Alasan Baik}
\newcommand{\theauthor}{Rivo Juicer Wowor}
\newcommand{\affiliation}{Kelompok 2B}

% \title{\textbf{\mytitle}}
% \author{\theauthor \\
        % \small{\affiliation}}
% \date{}

\usepackage{fancyhdr}
\fancypagestyle{plain}{%
  \fancyhf{}%
  \lhead{\footnotesize{\textbf{\mytitle}}}%
  % \rhead{\small{\textbf{Rivo Juicer Wowor (00000059635)}}}%
  \fancyfoot[R]{\thepage}%
  \fancyfoot[L]{\footnotesize{\theauthor \hspace{1pt} (\affiliation)}}
  \renewcommand{\headrulewidth}{0.4pt}% Line at the header invisible
  \renewcommand{\footrulewidth}{0.4pt}% Footer line not visible with 0pt
}


\pagestyle{plain}
\setlength{\parindent}{2em}
\renewcommand{\baselinestretch}{1.5}

\begin{document}
\begin{titlepage}
  \vspace*{\fill}
  \centering
  Assignment Pertemuan 6 \\
  Perjalanan Demokrasi di Indonesia \par
  \vspace{0.5cm}
  \LARGE{\textbf{\mytitle}} \par
  \vspace{0.5cm}
  \large{\theauthor} \par
  \small{\emph{\affiliation}} \par
  \vfill
\end{titlepage}
Seseorang pasti memiliki tujuan atau motif tertentu dalam melakukan
sebuah kejahatan. Hal tersebut juga berkaitan dengan definisi KBBI
sendiri yang menuliskan bahwa kejahatan berarti \emph{perilaku yang
bertentangan dengan nilai dan norma yang berlaku yang telah disahkan
oleh hukum tertulis}. Biasanya, tujuan seseorang melakukan kejahatan
dihubungkan terhadap kepentingan seorang individu atau sebuah kelompok.
Tapi terkadang, banyak kita temui orang-orang yang melakukan kejahatan
memiliki alasan seperti tidak punya uang untuk makan, untuk menghidupi
keluarga, dan lain-lain. Seperti salah satu kisah dari seorang pemulung
yang mencuri padi karena tidak mempunyai uang untuk makan. Dilansir dari
Media (2020), pemulung ini mengatakan bahwa ia tidak bisa mencukupi
kebutuhan keluarga karena banyak akses kampung ditutup oleh warga karena
wabah virus COVID-19 ini. Hal ini tentu saja membuat banyak orang dilema
ketika mendengar kisahnya. Tapi apakah hal yang ia perbuat terpuji?

Menurut saya orang yang mempunyai alasan seperti memang memiliki tujuan
yang baik, hanya saja tindakan yang ia lakukan merupakan tindakan yang
kurang tepat. Karena masih banyak cara dan pekerjaan halal yang dapat
dilakukan untuk mendapatkan uang. Hanya saja, kebanyakan orang ingin
mengambil jalan pintas seperti merampok dan mencopet. Selain itu, dengan
mempunyai alasan baik bukan berarti mereka bisa mendapatkan keringanan
hukum. Karena apapun alasannya jika dianggap salah oleh hukum, tetap
bersalah di mata hukum. Dan hal ini seringkali dilupakan oleh
orang-orang.

Tapi bukan berarti hanya pelaku kejahatan yang dimotivasi oleh
kemiskinan ini yang disalahkan. Karena seharusnya pemerintah memiliki
kewajiban dalam memelihara orang miskin dan anak terlantar, seperti yang
tertera dalam Undang-Undang Dasar 1945 pasal 34 ayat 1 yang menuliskan
bahwa \emph{``Fakir miskin dan anak-anak terlantar dipelihara oleh
negara''.} Jika seseorang yang miskin rela melakukan kejahatan demi
menafkahi hidup keluarganya, maka secara hukum hal ini juga menjadi
tanda bahwa pemerintah gagal dalam melakukan kewajibannya.

\hypertarget{refs}{}
\section*{Daftar Pustaka}
\begin{CSLReferences}{1}{0}
\leavevmode\vadjust pre{\hypertarget{ref-mediaKisahDiBalik2020}{}}%
Media, K. C. (2020). {Kisah di Balik Pemulung Curi Padi, Tak Punya Uang
untuk Makan dan Hidupi 5 Anggota Keluarga}. In \emph{KOMPAS.com}.
\url{https://regional.kompas.com/read/2020/04/23/15252001/kisah-di-balik-pemulung-curi-padi-tak-punya-uang-untuk-makan-dan-hidupi-5}

\end{CSLReferences}

\end{document}

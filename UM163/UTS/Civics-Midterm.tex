\PassOptionsToPackage{unicode}{hyperref}
\PassOptionsToPackage{hyphens}{url}
%
\documentclass[
  11pt,
  answers  
]{exam}
\usepackage[fleqn]{amsmath}
\usepackage[]{plex-otf}
\usepackage{iftex}
\usepackage[a4paper, margin=2.5cm]{geometry}
\usepackage{graphicx}
\usepackage[sorting=none, backend=biber]{biblatex}
% \ifPDFTeX
%   \usepackage[T1]{fontenc}
%   \usepackage[utf8]{inputenc}
%   \usepackage{textcomp} % provide euro and other symbols
% \else % if luatex or xetex
%   \usepackage{unicode-math}
%   \defaultfontfeatures{Scale=MatchLowercase}
%   \defaultfontfeatures[\rmfamily]{Ligatures=TeX,Scale=1}
% \fi
% Use upquote if available, for straight quotes in verbatim environments
\IfFileExists{upquote.sty}{\usepackage{upquote}}{}
\IfFileExists{microtype.sty}{% use microtype if available
  \usepackage[]{microtype}
  \UseMicrotypeSet[protrusion]{basicmath} % disable protrusion for tt fonts
}{}
\makeatletter
\makeatother
\usepackage{xcolor}
\IfFileExists{xurl.sty}{\usepackage{xurl}}{} % add URL line breaks if available
\IfFileExists{bookmark.sty}{\usepackage{bookmark}}{\usepackage{hyperref}}
\urlstyle{same} % disable monospaced font for URLs
\setlength{\emergencystretch}{3em} % prevent overfull lines
\providecommand{\tightlist}{%
  \setlength{\itemsep}{0pt}\setlength{\parskip}{0pt}}
\setcounter{secnumdepth}{-\maxdimen} % remove section numbering
\newlength{\cslhangindent}
\setlength{\cslhangindent}{1.5em}
\newlength{\csllabelwidth}
\setlength{\csllabelwidth}{3em}
\newlength{\cslentryspacingunit} % times entry-spacing
\setlength{\cslentryspacingunit}{\parskip}
\newenvironment{CSLReferences}[2] % #1 hanging-ident, #2 entry spacing
 {% don't indent paragraphs
  \setlength{\parindent}{0pt}
  % turn on hanging indent if param 1 is 1
  \ifodd #1
  \let\oldpar\par
  \def\par{\hangindent=\cslhangindent\oldpar}
  \fi
  % set entry spacing
  \setlength{\parskip}{#2\cslentryspacingunit}
 }%
 {}
\usepackage{calc}
\newcommand{\CSLBlock}[1]{#1\hfill\break}
\newcommand{\CSLLeftMargin}[1]{\parbox[t]{\csllabelwidth}{#1}}
\newcommand{\CSLRightInline}[1]{\parbox[t]{\linewidth - \csllabelwidth}{#1}\break}
\newcommand{\CSLIndent}[1]{\hspace{\cslhangindent}#1}
\ifLuaTeX
  \usepackage{selnolig}  % disable illegal ligatures
\fi
\renewcommand{\solutiontitle}{\noindent\textbf{Jawab:}\par\noindent}
\newcommand{\mytitle}{Civics (UM 163)}
\newcommand{\theauthor}{Rivo Juicer Wowor}
\newcommand{\affiliation}{Kelompok 2B}

% \title{\textbf{\mytitle}}
% \author{\theauthor \
		% \small{\affiliation}}
% \date{}

% \usepackage{fancyhdr}
\lhead{\footnotesize{\textbf{\mytitle}}}%
\rfoot{\thepage}%
\cfoot{}
\lfoot{\footnotesize{\theauthor \hspace{1pt} (\affiliation)}}
\pagestyle{headandfoot}

% \pagestyle{plain}
\setlength{\parindent}{2em}
\renewcommand{\baselinestretch}{1.5}
\hbadness=99999
\addbibresource{../../ref/library.bib}
\usepackage{enumitem}
\unframedsolutions

\begin{document}
	\begin{titlepage}
		\centering
		\vspace{2cm}
		\includegraphics[width=0.5\textwidth]{../../ref/logoUMN.png}\par\vspace{1cm}
		\vspace{1.5cm}
		\Large{Midterm Assignment} \par
		\vspace{1cm}
		\huge{\textbf{\mytitle}} \par
		\vspace{1.5cm}
		\Large{\theauthor} \par
		\emph{\affiliation} \par
		\vfill
		\today
		\end{titlepage}	

    \begin{questions}
      \question
      \begin{parts}
        \part Sebagai negara majemuk, apakah Indonesia dapat dikatakan sebagai negara bangsa? Jelaskan jawaban saudara dan kaitkan dengan artikel di atas! 
      \begin{solution}
        Negara bangsa merupakan sebuah negara yang masyarakat atau penduduknya sendiri mengganggap diri mereka sebagai suatu bangsa yang satu \cite[p~. 19]{daniels.pappContemporaryInternationalRelations1988}. Dan dari definisi tersebut, saya berpendapat bahwa Indonesia bisa dikatakan sebagai negara bangsa. Kita dapat melihat buktinya dari artikel berita diatas, dimana Pemerintah Kota Bogor memberikan izin IMB kepada GKI Yasmin yang sebelumnya sempat ditolak oleh warga. Tindakan Pemkot Bogor ini tentu saja berkaitan dengan nilai persatuan dan gotong royong yang ada pada Pancasila. Dan dari nilai persatuan tersebut kita dapat menyimpulkan bahwa adanya sebuah tujuan atau tindakan yang dilakukan masyarakat Indonesia; dalam hal ini Pemerintah Kota Bogor untuk mencapai suatu persatuan dalam keberagaman yang ada di Indonesia ini, sesuai dengan konsep negara bangsa tadi.
      \end{solution}

        \part Apakah mungkin mewujudkan Bhinneka Tunggal Ika di Indonesia? Jelaskan pendapat anda dan beri jalan keluar yang feasible!
      \begin{solution}
        Sangat mungkin, karena \emph{Bhinneka Tunggal Ika} sendiri merupakan semboyan dari dasar negara kita, yaitu Pancasila. Selain itu, tujuan utama Indonesia ketika memperjuangkan kemerdekaannya ialah menyatukan masyarakat Indonesia yang beragam budaya, suku, agama dan ras ini menjadi satu. Dan semangat persatuan tersebut seharusnya bisa dibawa hingga ke masa sekarang ini. 
        
        Tapi sekarang ini, kita dapat melihat banyaknya orang-orang yang mengedepankan politik Identitas ketimbang mengedepankan persatuan Indonesia. Dan tentunya hal tersebut harus kita berantas. Banyak hal yang dapat kita lakukan untuk menekan politik identitas tersebut, seperti mengajarkan nilai-nilai persatuan dan toleransi sejak dini, mencabut semua akar-akar radikalisme yang melahirkan politik identitas tersebut, dan memandang orang lain bukan dari latar belakangnya, tetapi sebagai sesama rakyat Indonesia yang telah merdeka.
      \end{solution}
      \end{parts}

      \pagebreak

      \question
      \begin{parts}
        \part Indonesia pernah memiliki "wawasan nusantara" sebagai visi atau politik nasional Indonesia. Jelaskan pemahaman anda mengenai konsep "wawasan nusantara" dan berikan tanggapan anda mengenai gerakan separatis seperti berita di atas, dari sudut pandang "wawasan nusantara" itu!
      \begin{solution}
        Dari yang saya pahami, \emph{Wawasan Nusantara} adalah suatu visi atau cara pandang bangsa Indonesia untuk menyatukan rakyat Indonesia yang terpisah di berbagai pulau menjadi kesatuan yang utuh dalam NKRI. Visi Wawasan Nusantara ini diciptakan untuk mempertahankan rasa persatuan yang sudah ada sejak Indonesia merdeka.
        
        Dari artikel tersebut, disebutkan bahwa empat prajurit TNI yang sedang bertugas di Pos Persiapan Koramil Kisor tewas dibunuh oleh 50 orang tak dikenal. Kemudian Sebby Sambom, Juru Bicara dari TPNPB-OPM, mengakui bahwa pihak Organisasi Papua Merdeka yang bertanggung jawab atas hal tersebut. Tindakan yang dilakukan oleh Organisasi Papua Merdeka tentunya merusak rasa solidaritas dan persatuan yang ada, karena mereka ingin melakukan aksi separatisme dan memisahkan diri dari Indonesia. Tentunya hal ini sangat bertentangan dengan visi \emph{Wawasan Nusantara} NKRI yang ingin menjamin persatuan dan kesatuan nasional dalam Negara Indonesia.
      \end{solution}

        \part Indonesia memiliki strategi untuk mencapai visinya tersebut, dan diberi nama "ketahanan nasional". Berikan upaya atau strategi yang bisa diambil pemerintah kita untuh mencegah maupun menindak segala tindakan disintegrasi!
      \begin{solution}
        Untuk menumpas segala tindakan disintegrasi, tentunya perlu upaya yang matang dari pemerintah Indonesia. Salah satu caranya adalah dengan pendidikan/pelatihan bela negara. Hal ini dilakukan untuk melatih putra-putri Indonesia dalam membela bangsa dan juga menumbuhkan jiwa nasionalisme yang ada di dalam diri mereka. Lalu, perlunya kerjasama antara masyarakat dan pemerintah dalam menumpas segala macam propaganda dan berita palsu yang disebarkan oleh pihak separatis ataupun dari sumber lain. Agar masyarakat Indonesia; terutama yang mudah dipengaruhi, bisa terhindarkan dari bahayanya aksi separatisme yang ada.
      \end{solution}
      \end{parts}
      
      \pagebreak

      \question
      \begin{parts}
        \part Berikan komentar anda terhadap artikel di atas dalam kaitan dengan konsep negara hukum \emph{(rechtstaat)} yang dianut Indonesia!
      \begin{solution}
        Negara hukum merupakan suatu konsep dimana dasar kekuasaan pemerintahan dan kehidupan masyarakatnya harus berlandaskan hukum. Oleh karena itu, masyarakat Indonesia; tidak peduli jabatan atau posisinya harus patuh oleh hukum undang-undang yang berlaku. Kita bisa melihat contohnya dari artikel diatas, dimana dua ibu rumah tangga nekat mencuri dagangan di dua toko kelontong. Pemilik toko tersebut menangkap ibu-ibu ini lalu melaporkannya ke polisi. Polisi juga langsung menahan keduanya dan memproses kasus tersebut.

        Dari contoh tersebut, kita bisa simpulkan bahwa apapun kejahatannya; baik kecil maupun besar, semuanya akan dipandang sama di mata hukum. Pelanggar hukum akan menerima hukuman yang sesuai dengan undang-undang yang berlaku, seperti kedua ibu tadi yang mencuri barang dagangan pada artikel diatas.
      \end{solution}

        \part Jelaskan pemahaman anda mengenai \emph{due process of law} dikaitkan dengan artikel di atas!
      \begin{solution}
        \emph{Due process of law} adalah suatu proses hukum yang adil dan tidak memihak. Dari artikel tersebut, kita dapat melihat dua ibu-ibu yang meskipun berstatus sebagai ibu rumah tangga, tetap ditahan dan diproses kasusnya oleh kepolisian. Hal ini membuktikan bahwa hukum tidak memandang status atau jabatan seseorang, semua dipandang sama dan menerima konsekuensi yang sama jika melanggar hukum tersebut.
      \end{solution}
      \end{parts}
    \end{questions}
    \nocite{sayutiKonsepRechtsstaatDalam2011,wijayanthiDueProcessLaw}
    \pagebreak
    \printbibliography

		
\end{document}

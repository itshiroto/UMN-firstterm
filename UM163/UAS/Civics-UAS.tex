\documentclass[11pt]{article}
\usepackage[a4paper, margin=3cm, left=4cm]{geometry}
\usepackage{microtype}
\usepackage{plex-otf}
% \usepackage{fontspec}
% \setmainfont{Times New Roman}

\usepackage[english, bahasai]{babel}
% \usepackage[
    % backend=biber,
    % sorting=none,
    % style=numeric
% ]{biblatex}
% \addbibresource{../../ref/library.bib}
\usepackage[numbers]{natbib}
\usepackage[autostyle]{csquotes}
\usepackage{hyperref}

% \usepackage{titling}
\usepackage{enumitem}
\usepackage{graphicx}
\usepackage{setspace}

% \usepackage{lipsum}

\renewcommand{\baselinestretch}{1.5}
\setlength{\parskip}{1em}

\usepackage{titlesec}
\titlelabel{\thetitle.\quad}
% \titleformat*{\section}{\large\bfseries}
% \titleformat*{\subsection}{\normalsize\bfseries}
% \titlespacing{\section}{0pt}{\parskip}{-\parskip}
\titlespacing{\subsection}{0pt}{\parskip}{-\parskip}

\newcommand{\mytitle}{UM163 - Civics}
\newcommand{\theauthor}{Rivo Juicer Wowor}
\newcommand{\affiliation}{Kelompok 2B}

\title{Korupsi di Indonesia dan Nilai 5C sebagai Landasan Bermoral untuk Mencegahnya}
\author{\theauthor \\ 00000059635 \\ {\small\affiliation}}
\date{}

\usepackage{fancyhdr}
\fancyhf{}%
\lhead{\footnotesize{\textbf{\mytitle}}}%
\rhead{}
\fancyfoot[R]{\thepage}%
\fancyfoot[L]{\footnotesize{\theauthor \hspace{1pt} (\affiliation)}}
 
\pagestyle{fancy}
\thispagestyle{plain}

\begin{document}
\maketitle
\section{Pendahuluan}
    \subsection{Latar Belakang}
    Korupsi merupakan suatu masalah yang sangat besar dan sering terjadi di negara
    Indonesia. Jika anda melakukan pencarian di situs \emph{Google} dan mengetik
    kata kunci "korupsi", anda akan menemukan begitu banyaknya kasus korupsi yang
    terjadi hampir setiap hari. Menurut statistik dari \emph{Indonesian Corruption Watch}
    yang dikutip oleh \emph{Tempo.co}, terjadi peningkatan jumlah penindakan kasus 
    korupsi sebesar 40 kasus pada Semester 1 2021 jika dibandingkan dengan tahun 2020. 
    Serta meningkatnya nilai kerugian negara akibat korupsi sebesar 47,6\% dalam 
    setahun terakhir. Dan hal ini tentu saja sangat berbahaya jika dibiarkan terus
    menerus; bukan hanya dalam aspek ekonomi saja tapi dalam aspek sosial dan
    kepercayaan \citep{setiadiKORUPSIDIINDONESIA2018}.
    Generasi muda Indonesia akan menganggap korupsi sebagai suatu \emph{"budaya"}
    sehingga menghasilkan anak muda yang tidak jujur dan korup. Lalu masyarakat
    akan kehilangan kepercayaannya kepada pemerintah karena mereka akan berpikir
    bahwa semua pejabat pemerintahan itu korup, dan bahkan menciptakan masyarakat 
    yang kacau.

    Ada banyak faktor yang menyebabkan terjadinya suatu kasus korupsi ini. Tapi 
    salah satu faktor yang paling berpengaruh adalah kurangnya kesadaran moral 
    oleh para pelaku koruptor di Indonesia. Bisa dibilang koruptor memiliki latar
    belakang pendidikan yang tinggi, tetapi pendidikan moralnya masih dibawah 
    tingkat dasar. Sehingga salah satu cara yang dapat dilakukan untuk mencegah
    terjadinya tindak korupsi ini adalah dengan memperbaiki kesadaran dan pendidikan
    moral pada seluruh masyarakat Indonesia. 

    Pendidikan moral ini kita bisa terapkan menggunakan nilai 5C dari Kompas
    Gramedia untuk membantu masyarakat dalam memahami serta mendalami moralitas 
    seseorang. Nilai 5C merupakan lima sifat yang diwariskan oleh dua pendiri Kompas
    Gramedia, yaitu bapak Petrus Kanisius Ojong serta bapak Jakob Oetama. Dan 
    dipegang teguh oleh Kompas Gramedia hingga saat ini. Nilai-nilai ini ditetapkan
    sebagai pedoman dalam berperilaku, berpikir, serta bertindak. Dan dari kelima
    nilai ini, kita dapat implementasikan ke dalam pendidikan anti korupsi untuk
    masyarakat Indonesia dalam mencegah terjadinya tindakan korupsi.

    \subsection{Rumusan Masalah}
    \begin{enumerate}
        \item Apa itu korupsi dan mengapa korupsi merajalela di Indonesia?
        \item Apa saja faktor-faktor yang dapat mengakibatkan seseorang melakukan
              korupsi?
        \item Mengapa faktor moral itu sangat penting dalam mencegah tindakan
              korupsi?
        \item Apakah nilai 5C Kompas bisa dipakai untuk mencegah korupsi 
              dan bagaimana penerapannya?
    \end{enumerate}

\newpage
\section{Pembahasan}
\subsection{Pengertian serta bahaya Korupsi}
Untuk membahas korupsi lanjut, kita harus memahami apa arti dari kata \emph{korupsi}
itu sendiri. Menurut KBBI Daring Edisi 3, korupsi adalah suatu penyelewenagan 
atau penyalahgunaan uang negara (perusahaan, dan lainnya) untuk keuntungan diri sendiri 
atau orang lain. Korupsi juga merupakan suatu kata yang berasal dari kata
\emph{korup} yang berarti busuk, buruk dan suka menerima uang sogok \citep{purwadarmintaKamusUmumBahasa1982} sehingga
dapat dikatakan bahwa korupsi merupakan suatu perbuatan buruk seperti penggelapan uang,
penerimaan uang sogok, dan lainnya \citep{setiadiKORUPSIDIINDONESIA2018}.

Beberapa ahli juga mempunyai definisi atau arti dari korupsi berdasarkan pendapat mereka.
Yang pertama adalah dari Shleifer et al. dimana ia berpendapat bahwa korupsi adalah
\emph{the sale by government official of government property for personal gain}
\citep{shleiferCorruption1993}, atau suatu penyalahgunaan atau penjualan properti
negara oleh pejabat untuk keuntungan dirinya sendiri. Lalu ada juga pendapat dari
Svensson dimana ia mengatakan \emph{"public corruption is the misuse of public
office for private gain"} atau penyalahgunaan jabatannya dalam pemerintahan untuk
kepentingan pribadi \cite[pg. 20]{svenssonEightQuestionsCorruption2005}.
Svensson juga berpendapat bahwa korupsi merupakan suatu refleksi dari keadaan
legal, ekonomi, budaya, dan politik yang ada di suatu negara. Sehingga dari 
semua pengertian tersebut dapat dipahami bahwa korupsi merupakan suatu tindakan
penyalahgunaan hak kekuasaan pejabat untuk menguntungkan diri sendiri dengan cara
menggelapkan uang negara, menjual aset negara, melakukan penyogokan, dan lainnya.

Terdapat tiga jenis korupsi dalam lingkungan demokrasi negara menurut Arvind K. Jain
\citep{jainCorruptionReview2001},
yaitu:
\begin{enumerate}
    \item \textbf{Grand Corruption} \\
          \emph{Grand Corruption} biasanya mengarah kepada suatu perbuatan yang dilakukan
          oleh pejabat politik dimana mereka mengekspolitasi kekuasaan yang dimilikinya
          untuk membuat suatu kebijakan ekonomi yang menguntungkan dirinya. Contohnya
          ketika Menteri Kesehatan bekerja sama dengan petinggi-petinggi negara untuk
          mengalokasikan sebagian sumber daya dari  sistem rumah sakit nasional ke perusahaannya 
          sendiri. Sehingga menteri tadi ini dapat memanipulasi harga obat dan alat medis
          lainnya untuk menguntungkan dirinya sendiri \citep{transparencyinternationalGrandCorruptionOur}. 
          Tipe korupsi ini biasanya sulit untuk dilacak karena tipe ini menggunakan
          undang-undang dan kebijakan nasional lainnya dan disahkan secara diam-diam 
          sehingga masyarakat publik tidak mengetahui tentang kebijakan ini.
          
    \item \textbf{Bureaucratic Corruption} \\ 
          \emph{Bureaucratic Corruption} merupakan suatu tipe korupsi dimana pejabat
          - pejabat yang telah dipilih oleh rakyat memiliki suatu persetujuan dengan atasan atau publik
          untuk keuntungannya. Biasanya tipe ini juga bisa disebut sebagai \emph{petty corruption}
          atau korupsi kecil-kecilan, karena tipe korupsi ini membuat publik untuk membayar atau
          menyuap pejabat tersebut untuk menerima suatu layanan yang seharusnya bisa diterima
          secara gratis atau lebih murah, atau bahkan layanan yang seharusnya tidak ada,
          bisa dilakukan dengan cara disuap terlebih dahulu. Contohnya seperti ketika membuat
          SIM, ketimbang mengikuti ujian SIM; orang-orang akan memilih untuk menyuap atasan polisi
          atau menggunakan calo untuk mendapatkan SIM.

    \item \textbf{Legislative Corruption} \\
          Legislative Corruption adalah suatu tindakan korupsi dimana seorang calon pejabat atau
          pejabat yang sudah terpilih menggunakan kekayaannya untuk memanipulasi hasil pemilihan atau keputusan.
          Tipe ini juga bisa disebut sebagai \emph{money politics}. Pejabat tersebut dapat menyogok
          suatu kelompok tertentu untuk mengesahkan kebijakannya atau untuk memberinya kemenangan dalam suatu
          pemilihan. Tipe ini bisa juga berkaitan dengan tindakan \emph{vote-buying} dimana suatu pemimpin daerah
          atau pejabat menyuap atasannya agar bisa masuk atau bahkan dipilih kembali dalam suatu pemilihan.
\end{enumerate}

Dari tipe-tipe tersebut, tentunya kita menyadari bahwa korupsi itu sangat berbahaya
bagi suatu negara, terkhususnya Indonesia. Dan bahkan menurut
\citeauthor*{setiadiKORUPSIDIINDONESIA2018}, korupsi bukan hanya berbahaya
untuk aspek ekonomi negara saja, tapi bisa mempengaruhi aspek sosial yang ada
di masyarakat. Karena jika korupsi dalam suatu negara atau masyarakat itu merajalela
dan menjadi \emph{tontonan} rakyat setiap hari, maka besar kemungkinannya bahwa
masyarakat akan menjadi kacau karena sistem sosial yang ada di negara tersebut
tidak dapat berlaku dengan baik. Hal ini dikarenakan setiap individu hanya akan
mementingkan diri siri sendiri dan merusak persaudaraan yang tulus, karena melihat 
orang-orang yang ada di kursi wakil rakyat sudah berbuat demikian.

Selain itu, korupsi juga dapat memberikan efek negatif pada generasi muda.
Karena sudah termakan oleh kasus korupsi yang muncul hampil setiap hari,
anak-anak muda akan berkembang menjadi orang yang terbiasa dengan sifat yang tidak jujur serta tidak bertanggung
jawab. Dan jika hal ini terjadi, tentunya akan mengubah masa depan dirinya, dan 
juga negaranya. Dan yang terakhir adalah korupsi juga dapat menghasilkan pemerintah
yang tidak sah di mata publik, dan masyarakat akan kehilangan kepercayaannya terhadap
pejabat dan juga orang-orang penting lainnya yang ada di kursi pemerintahan. Dan
pada akhirnya, masyarakat akan menjadi tidak patuh terhadap pemerintah lagi.
Tindak korupsi kecil-kecilan seperti \emph{money politics}, kecurangan dalam pemilu,
dan lainnya juga dapat menyebabkan demokrasi yang ada di suatu negara. Karena 
jika pelaku korupsi tersebut lolos menjadi penguasa di suatu daerah, tentunya
dia akan melakukan aksi korupsi yang lebih besar lagi di masyarakat.

\subsection{Mengapa korupsi merajalela di Indonesia}
Indonesia merupakan suatu negara yang memiliki tingkat korupsi yang cukup tinggi 
di dunia. Data dari \citeauthor{transparencyinternationalCorruptionPerceptionsIndex2020}
menunjukkan bahwa Indonesia berada di peringkat 102 di dunia dari negara lainnya,
dengan skor Indeks Persepsi Korupsi (CPI) yang dimiliki Indonesia sebesar 37. 
Indeks Persepsi Korupsi sendiri merupakan suatu data yang memberikan peringkat 
terhadap 180 negara atau wilayah khusus berdasarkan tingkat korupsi di sektor publik
dan dirasakan oleh para ahli maupun pebisnis di negara tersebut. Skala penilaiannya
sendiri dimulai dari 0 yang berarti sangat korup, hingga 10 yang berarti sangat
bersih. Skor sebesar 37 yang dimiliki oleh Indonesia ini menandakan bahwa masih
banyak korupsi yang terjadi di negara ini. 

Dan hal ini tidak terjadi hanya di Indonesia saja, tapi juga beberapa negara yang
ada di dunia seperti Myanmar, Thailand, dam Vietnam. Dan negara-negara tersebut
memiliki beberapa karakteristik yang menjadi persamaan sehingga kita bisa 
mendapatkan petunjuk tentang bagaimana sebuah negara bisa memiliki tingkat korupsi 
yang tinggi \citep{svenssonEightQuestionsCorruption2005}.

Yang pertama, rata-rata negara yang memiliki tingkat kasus korupsi yang cukup
tinggi biasanya adalah negara berkembang dan/atau negara yang sedang dalam 
proses transisi. Dan beberapa ahli yang berargumen bahwa kualitas suatu pemerintahan
itu dibentuk dari faktor ekonomi negaranya, dan hal ini yang menyebabkan munculnya
tindakan korupsi. Singkatnya, sebuah pemerintah atau lembaga dapat berkembang dalam 
menanggapi tingkat pendapatan dan kebutuhan daerah yang berbeda
\citetext{\citealp{demsetzTheoryPropertyRights1974, lipsetPoliticalManSocial1960}
dalam \citealp{svenssonEightQuestionsCorruption2005}}. Negara-negara tersebut
juga biasanya memiliki tingkat pertumbuhan ekonomi atau GDP per kapita 
yang rendah. Penelitian yang dilakukan oleh \citet{paolomauroEconomicIssuesNo1997}
menunjukkan bahwa korupsi memiliki hubungan terbalik yang berkaitan dengan tingkat
pertumbuhan ekonomi sebuah negara. Penelitian lebih lanjut menunjukkan bahwa jika
Ineks Persepsi Korupsi yang dimiliki oleh suatu negara itu naik satu saja,
maka tingkat pertumbuhan ekonomi dan investasi yang dimiliki negara tersebut akan 
naik sebesar lebih dari 4 persen serta pertumbuhan tahunan GDP per kapita meningkat
lebih dari setengah persen. Sehingga dapat dikatakan jika banyak korupsi yang terjadi
di negara tersebut, maka akan menimbulkan efek penurunan pada ekonomi negaranya.

Lalu, kita juga bisa melihat dari sisi sejarah negara tersebut. Rata-rata negara
yang memiliki jumlah kasus korupsi tinggi biasanya merupakan negara bekas peninggalan
pemerintahan kolonial seperti Prancis dan Belanda. \citet*{laportaQualityGovernment1999}
yang dikutip dari \citet*{svenssonEightQuestionsCorruption2005} menegaskan bahwa 
identitas dari penjajah termasuk sistem hukumnya itu berpengaruh terhadap koloni 
atau negara yang dijajah tersebut. Contohnya, menurut mereka negara bekas penjajahan
Prancis atau negara-negara berideologi Sosialis biasanya membuat banyak peraturan,
dan peraturan atau regulasi yang terlalu banyak mengarah kepada tindakan korupsi. 
Kemudian, faktor pendidikan juga berpengaruh terbalik dalam jumlah kasus korupsi di suatu
negara. Maksudnya adalah pendidikan tinggi dapat menciptakan masyarakat yang
teredukasi sehingga mereka dapat berpikir secara kritis. Banyak orang mungkin berkata
bahwa "korupsi banyak dilakukan oleh orang-orang yang memiliki gelar tinggi". Tapi
menurut saya kita tidak boleh hanya melihat dari sisi gelarnya saja, melainkan 
kita harus melihat dari kualitas institusi pemerintahannya.
Menurut \citet*{vanderbrugErosionPoliticalTrust2007} yang dikutip dari 
\citet*{agerbergCurseKnowledgeEducation2019}, perilaku politik dari seseorang
yang berpendidikan tinggi bisa saja terpengaruh oleh institusi pemerintahan yang
bekerja secara buruk. Sehingga hubungan pendidikan dengan korupsi seharusnya
saling berkebalikan. Dari artikel yang sama, \citeauthor{agerbergCurseKnowledgeEducation2019}
juga berpendapat bahwa orang-orang yang teredukasi seharusnya dapat berpikir kritis 
dan bisa berpengaruh dalam menyuarakan tindak korupsi ini.
Dari keempat poin in, saya dapat menyimpulkan bahwa kualitas suatu
lembaga pemerintahan diukur dari faktor ekonomi negara tersebut, serta bagaimana
kondisi pemberdayaan sumber daya manusianya. Dan kualitas lembaga pemerintahan
ini yang nantinya akan menjadi salah satu faktor yang mempengaruhi seseorang
dalam melakuan korupsi.

Sejauh ini, saya menjelaskan penyebab terjadinya korupsi ketika dilihat dari
sisi karakteristik negara yang memiliki tingkat kasus korupsi yang tinggi. Tapi,
kita juga dapat melihat dari beberapa aspek lainnya yang ada. Yang pertama adalah 
penyebab terjadinya korupsi dilihat dari aspek pemerintahan. Banyak orang yang
berargumen bahwa salah satu faktor terbesar mengapa di Indonesia terjadi banyak
sekali kasus korupsi dikarenakan sanksi hukuman koruptor yang terlalu ringan di 
dalam undang-undang. Ketua Komisi Pemberantasan Korupsi (KPK) Firli Bahuri menyatakan
bahwa \emph{"Orang melakukan korupsi karena ancaman hukumannya bahkan vonisnya rendah"}
\citep{okezoneKetuaKPKKorupsi2020}. Dan juga
Koordinator Divisi Hukum dan Monitoring Peradilan ICW, Emerson Yuntho mengatakan
bahwa 
\begin{quote}
    \emph{"Pasal yang sering didakwakan kepada koruptor adalah Pasal 3 Undang-Undang
    Tindak Pidana Korupsi dengan ancaman hukuman minimal 1 tahun penjara dan
    maksimal seumur hidup,"}\citep{saputriKenapaHukumanKoruptor2015}
\end{quote}
Majelis Hakim di Indonesia lebih sering menggunakan pasal tersebut ketika menghukum
koruptor-koruptor di Indonesia, sehingga hukuman yang biasa diterima oleh koruptor
rata-rata 2 tahun penjara saja. Hal ini yang dirasa oleh banyak orang kurang
memberikan efek jera kepada koruptor-koruptor tersebut. Kemudian hal ini juga
ditambah oleh kurangnya transparansi data yang diberikan oleh pemerintah. Contohnya
dalam BUMN, salah satu faktor yang menyebabkan sering terjadinya korupsi adalah
data keuangan BUMN yang sangat sulit diakses \citep{alaidrusTransparansiMinimKepentingan}.
Akibatnya, data keuangan yang tertutup tersebut membuka celah untuk korupsi. Hal
ini juga ditambah dengan lemahnya pengawasan korupsi dari lembaga eksternal maupun
internal di Indonesia. Contohnya seperti kejadian beberapa tahun terakhir dimana
fungsi KPK dilemahkan serta KPK yang bermain tebang pilih yang menandakan kurangnya
pengawasan dari lembaga internal pemerintahan. 

Kemudian kita dapat lihat dari aspek pribadi pelaku-pelaku korupsi tersebut. 
Salah satu penyebab terjadinya korupsi adalah seseorang bekerja hanya bertujuan
ingin mendapatkan uang saja serta mengakibatkan munculnya sifat rakus
serta konsumtif pada orang tersebut. Sifat-sifat ini jika dibiarkan terus menerus 
maka akan mengakibatkan tindakan korupsi. Lalu kita juga bisa melihat dari moralitas
orang tersebut. Seperti yang telah disebutkan tadi, pelaku-pelaku koruptor mungkin 
memiliki gelar pendidikan yang tinggi-tinggi. Akan tetapi, banyaknya pengetahuan
yang dimilikinya tak sebanding dengan rendahnya nilai moral yang ada di dalam dirinya.
Sehingga menyebabkan munculnya sifat kurang peduli, individualisme, serta tidak jujur
yang akhirnya berimbas pada terjadinya aksi korupsi. 

\subsection{Implementasi Nilai 5C untuk mencegah korupsi}
Dari analisa yang telah dilakukan, didapatkan bahwa ternyata banyak hal yang
dapat mengakibatkan terjadinya korupsi ini. Banyak faktor-faktor yang mendorong
seseorang melakukan korupsi seperti lemahnya kualitas institusi pemerintahan,
kurangnya pengawasan dari lembaga eksternal maupun internal negara, dan juga
faktor pendidikannya. Tapi dari poin-poin tersebut, saya dapat menyimpulkan
bahwa faktor terpenting seseorang akhirnya melakukan korupsi itu dilihat dari
kurangnya moralitas pelaku korupsi ini. Ilmu yang tinggi tetapi tidak disertai
dengan pendidikan moral yang baik justru akan membuat seseorang berperilaku anti
sosial serta rakus hingga akhirnya menyebabkan orang tersebut melakukan korupsi.

Ada banyak cara yang dapat dilakukan untuk memperbaiki moralitas masyarakat sejak
dini, salah satunya adalah dengan menggunakan nilai 5C dari Kompas Gramedia untuk
menciptakan individu yang terdidik dan juga memiliki moral yang tinggi. Tapi
sebelumnya, kita perlu tahu apa definisi serta isi dari nilai 5C ini sendiri. 
Nilai 5C adalah nilai-nilai keutamaan perusahaan Kompas Gramedia yang terdiri 
dari lima sifat yaitu:

\begin{enumerate}
    \item \textbf{Caring} \\
          Nilai Caring adalah suatu nilai yang didasarkan pada filosofi
          \emph{Humanisme Transendental} yang memiliki arti berperi kemanusiaan,
          berdasarkan keyakinan terhadap Tuhan Yang Maha Kuasa. Dari nilai ini 
          akan menumbuhkan rasa peduli sehingga orang yang menanamkan nilai ini 
          bisa saling peduli terhadap sesama baik di lingkungan pekerjaan maupun 
          masyarakat.
          
          Rasa peduli ini juga yang akan menjadi salah satu kunci 
          dalam mencegah terjadinya tindakan korupsi. Karena berdasarkan analisa
          sebelumnya, keinginan korupsi itu timbul dari sifat tidak peduli seseorang.
          Sehingga nilai Caring ini dapat digunakan untuk mencegah hal tersebut

    \item \textbf{Credible} \\
          Nilai Credible ini didasarkan pada suatu pemahaman bahwa manusia yang
          bekerja selalu berdimensi sosial, menuntuk interaksi timbal balik dengan
          lingkungannya. Sehingga jika seseorang menanam nilai ini dalam dirinya
          dan bekerja secara ikhlas, disiplin, konsisten, serta profesional, maka
          Ia akan menjadi orang yang dapat dipercaya dan juga dapat diandalkan
          oleh orang lain. 

          Nilai ini memiliki kaitan yang erat dengan sifat anti korupsi, dimana
          korupsi terjadi karena tumbuh rasa tidak ikhlas, serta rakus. Oleh karena
          itu, nilai ini dapat dipakai juga untuk mencegah munculnya tindakan korupsi
          dalam kehidupan seseorang.

    \item \textbf{Competent} \\
            Nilai ini memiliki pemahaman bahwa manusia yang bekerja harus selalu 
            bisa berkembang serta mengembangkan dirinya untuk memberikan hasil yang
            terbaik bagi dirinya sendiri, dan juga lingkungan sekitarnya.

            Seseorang yang kompeten memiliki kemungkinan yang sangat kecil untuk
            melakukan aksi korupsi, karena orang yang berkompeten tinggi akan 
            menggunakan kemampuannya dengan maksimal untuk menyelesaikan tugasnya
            dan juga tidak terfokus kepada penghasilan dan material saja.

    \item \textbf{Competitive} \\
            Nilai Competitive merupakan nilai yang menekankan pada keberanian
            dalam menghadapi tantangan. Nilai ini juga menunjukan kecerdasan mental
            \emph(Adversity Quotient) yang dapat mengubah ancaman yang ada menjadi
            peluang, untuk dapat selalu berkembang dan juga berfokus pada daya
            saing. 

            Nilai Competitive ini dapat kita pakai untuk menjadi pribadi yang 
            aktif serta mau terus berjuang. Dengan begitu, orang yang kompetitif 
            tentunya akan berusaha keras dalam pekerjaannya untuk menjadi yang
            terbaik. Sifat mau bekerja keras serta ingin menjadi yang terbaik ini
            sangat terhubung 

    \item \textbf{Customer Delight} \\
            Nilai Customer Delight adalah nilai yang menjunjung tinggi prinsip
            memenangkan hati pelanggan dengan memberikan pelayanan yang melebihi 
            ekspetasinya. Nilai ini memiliki hubungan yang erat dengan nilai 
            Competitive dan Credible dimana seseorang harus bekerja keras demi 
            menyenangkan konsumen atau pelanggannya dengan cara memberikan 
            pelayanan yang terbaik. 

            Nilai ini dapat dipakai juga untuk menumbuhkan sifat anti korupsi 
            pada masyarakat. Karena dalam diri orang yang menanamkan nilai ini,
            Ia dapat mengaplikasikan semua pengetahuannya ke dalam pekerjaannya 
            untuk memberikan yang terbaik kepada orang lain. Sehingga keinginan
            untuk melakukan korupsi justru akan hilang karena perbuatan korupsi 
            itu justru berkebalikan dengan tujuan dari nilai ini.
\end{enumerate}

Setelah memahami kelima nilai tersebut serta hubungannya dengan sifat anti korupsi,
maka kita dapat menanamkannya ke dalam kehidupan masyarakat dan mengaplikasikan
nilai-nilai tersebut dalam kehidupan sehari-hari. Karena faktor moral merupakan
pemicu dari tindakan korupsi dalam masyarakat, dan lahir dari tindakan-tindakan kecil
yang dilakukan oleh seseorang. Sehingga untuk memperbaiki faktor moral tersebut
diperlukannya landasan dalam menjalankan kehidupan seperti nilai-nilai 5C tersebut.

\newpage
\section{Penutup}
    \subsection{Kesimpulan}
    Korupsi merupakan salah satu masalah yang sangat besar dan belum teratasi
    dengan maksimal di Indonesia. Data dari \emph{Corruption Perceptions Index}
    juga menunjukkan betapa buruknya usaha penanggulangan korupsi di Indonesia ini.
    Mulai dari sanksi hukum koruptor yang dinilai terlalu ringan hingga lemahnya
    pendidikan moralitas yang ada di dalam masyarakat sehingga mengakibatkan tumbuhnya
    akar masalah korupsi dalam diri seseorang tersebut. 

    Untuk mencegah hal tersebut serta meningkatkan kualitas pendidikan moralitas
    di Indonesia, maka diperlukan semacam pedoman dalam masyarakat untuk dapat 
    mengerti mengenai moralitas serta menanamkan dalam kehidupan sehari-harinya.
    Nilai 5C Kompas Gramedia dapat menjadi salah satu kandidat dalam menjadi pedoman
    bermoral dalam masyarakat. Karena nilai 5C memiliki nilai-nilai yang berhubungan
    erat dengan sikap anti-korupsi sehingga kedepannya masyarakat dapat memiliki
    nilai moral yang tinggi serta sikap jujur dan peduli terhadap sesamanya untuk
    mencegah terjadinya korupsi ini. 
    
    \subsection{Saran}
    Berkaitan dengan masalah korupsi dan cara pencegahannya, saya menyarankan beberapa
    hal untuk diperhatikan dan dipertimbangkan seperti berikut ini:
    \begin{itemize}
        \item Menekankan serta menanamkan nilai-nilai 5C dalam kehidupan sehari-hari,
              bukan hanya di lingkungan Universitas Multimedia Nusantara dan Kompas
              Gramedia saja, tapi di seluruh Indonesia.
        \item Memberikan edukasi serta \emph{workshop} mengenai nilai 5C di sektor
              pendidikan dasar dan menengah 
    \end{itemize}
    Selain itu, penulis juga mengharapkan kritik dan saran dalam penulisan makalah 
    ini di kemudian hari.

\newpage
% \printbibliography[title=Daftar Pustaka]
\nocite{agerbergCurseKnowledgeEducation2019, alaidrusTransparansiMinimKepentingan, badanpusatpengembangandanpembinaanbahasaKorupsi, demsetzTheoryPropertyRights1974, dwiputriantiMEMAHAMISTRATEGIPEMBERANTASAN, firdausiJatuhBangunLembaga2017, izzudinAlasanKenapaKorupsi, jainCorruptionReview2001, javierICWAngkaPenindakan2021, laportaQualityGovernment1999, lipsetPoliticalManSocial1960, okezoneKetuaKPKKorupsi2020, paolomauroEconomicIssuesNo1997, purwadarmintaKamusUmumBahasa1982, saputriKenapaHukumanKoruptor2015, setiadiKORUPSIDIINDONESIA2018, shleiferCorruption1993, suparmo2016transformasi, svenssonEightQuestionsCorruption2005, syariefKorupsiKolektifKorupsi2018, transparencyinternationalCorruptionPerceptionsIndex2020, transparencyinternationalGrandCorruptionOur, vanderbrugErosionPoliticalTrust2007}
\bibliographystyle{plainnat}
\bibliography{../../ref/library-legacy}
\end{document}

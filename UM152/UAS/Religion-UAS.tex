\documentclass[12pt, answers]{exam}
\usepackage[a4paper, hmargin=0.5in, vmargin=1in]{geometry}
% \usepackage[bahasa]{babel} 
\usepackage[autostyle]{csquotes}
% \usepackage[fixlanguage]{babelbib}

\usepackage[backend=biber, sorting=none, sortlocale=auto]{biblatex}
\addbibresource{../../ref/library.bib}
\nocite{
    aarnioChallengesPackagingWaste2008,
    albertusKonsumerismeTerhadapGawai2020,
    chakrabartyHumanSciencesClimate2020,
    GlobalEwasteGeneration,
}

\usepackage{fontspec}
\setmainfont{Times New Roman}

\renewcommand{\baselinestretch}{1.5}

\renewcommand{\solutiontitle}{
    \noindent\textbf{Jawab:}\par\noindent
}

\newcommand{\mytitle}{UM152 - Religion}
\newcommand{\theauthor}{Rivo Juicer Wowor}
\newcommand{\affiliation}{00000059635}

\lhead{\footnotesize{\textbf{\mytitle}}}%
\rfoot{\thepage}%
\cfoot{}
\lfoot{\footnotesize{\theauthor \hspace{1pt} (\affiliation)}}
\pagestyle{headandfoot}
\unframedsolutions

\begin{document}
    \begin{questions}
        \question
        \begin{parts}
            \part Sebutkan serta berikan penjelasan 6 unsur agama!
            \part Manusia mendekati Tuhan dengan simbolisme melalui ritual.
                  Apakah simbolisme yang terdapat di agama yang Anda anut
                  berhasil mentransformasi Anda menuju arah yang baik? Berikan 
                  contoh kongkret!
        \end{parts}

        \begin{solution}
        \begin{parts}

        % Menurut Prof. Koentjaraningrat, setiap agama memiliki enam
        % unsur yakni:
        % - Religiositas
        % - Sistem kepercayaan
        % - Ritus
        % - Organisasi
        % - Pengamalan
        % - Fenomena kelompok
        \part \begin{enumerate}
            \item \textbf{Religiositas} \\
                Religiositas adalah suatu pengalaman ketergantungan manusia
                kepada Hyang Ilahi, dan merupakan inti dari agama-agama. 
                Karena jika suatu agama kehilangan unsur religiositasnya,
                maka agama tersebut dapat merasakan kekeringan.

            \item \textbf{Sistem Kepercayaan} \\
                Sistem kepercayaan adalah suatu ajaran yang dikembangkan dan 
                diberi penjelasan sehingga bisa menjadi bentuk sistem
                kepercayaan. Bentuk ini memiliki peran sebagai pandangan hidup
                bersama bagi umat yang menganutnya.
        
            \item \textbf{Ritus} \\
                Ritus merupakan suatu bentuk pengungkapan iman manusia kepada
                Hyang Ilahi. Ritus juga bisa disebutkan sebagai suatu tata cara
                beribadah yang bersifat seremonial dalam suatu agama.
            
            \item \textbf{Organisasi} \\
                Organisasi adalah suatu kesatuan atau susunan dalam kelompok
                yang memiliki pemimpin, pengikut, serta kewajiban yang diatur
                bersama-sama. Dalam agama, organisasi dibuat untuk mengatur,
                memelihara, serta menjaga keharmonisan umat beragama.

            \item \textbf{Pengamalan} \\
                Pengamalan adalah salah satu bentuk perwujudan iman seorang
                manusia dengan menunjukkan kepeduliannya terhadap sesama manusia
                dan alam. Pengalaman dapat juga dikatakan sebagai berbagi berkat
                yang telah dialami, serta menghadirkan kebaikan Hyang Ilahi
                ke dalam kehidupan sehari-hari.

            \item \textbf{Kelompok} \\
                Komunitas kurang lebih mirip dengan organisasi, hanya saja
                komunitas mencakup ranah yang lebih kecil ketimbang organisasi.
                Komunitas juga bertujuan untuk menyatukan diri umatnya dengan
                Hyang Ilahi. Komunitas memiliki norma-norma yang perlu ditaati
                bersama, serta punya sistem hierarki yang mengatur keterlibatan
                individu dalam kelompok, sehingga setiap anggota memiliki peran 
                serta tugasnya masing-masing sesuai dengan kemampuannya.
        \end{enumerate}
        
        % - Tentu saja
        % - Seperti ibadah, berdoa, serta melayani.
        %   - Berdoa: Salah satu cara untuk menenangkan diri, serta salah satu
        %             media untuk mempererat hubungan diri sendiri dengan Tuhan
        %   - Ibadah: Menguatkan hubungan antara umat beragama, dan mengajarkan
        %             norma-norma yang sangat bermanfaat untuk bertumbuh, baik
        %             dalam sosial maupun iman
        %   - Pelayanan: Mengajarkan diri untuk tetap rendah hati dan belajar
        %                berkembang secara jasmani dan rohani.

        \part Dalam Kekristenan, ada beberapa simbol serta ritual yang berhasil
        mentransformasi diri saya menuju arah yang lebih baik lagi.
        
        Yang pertama adalah \textbf{Doa}. Menurut saya, doa merupakan salah satu cara
        yang paling ampuh untuk menenangkan diri ketika dalam suatu masalah.
        Karena dengan doa, kita mengungkapkan semua isi hati kita kepada Tuhan;
        termasuk mengutarakan semua masalah yang kita hadapi. Selain itu, doa
        juga salah satu media yang bisa mempererat hubungan diri kita sendiri 
        dengan Tuhan. 

        Kedua adalah \textbf{Ibadah}. Ibadah adalah suatu kegiatan yang memiliki tujuan
        bukan hanya mendekatkan diri dengan Tuhan, tapi juga dengan orang-orang
        seiman dengan kita. Serta selama ibadah, komunitas dalam gereja saya juga 
        mengajarkan norma-norma yang bermanfaat untuk menumbuhkan jiwa sosial
        maupun iman kita terhadap Tuhan.

        Yang terakhir dan yang paling berpengaruh untuk saya adalah \textbf{Pelayanan}.
        Selama 8 tahun, saya diajarkan banyak hal dalam pelayanan. Yang pertama
        adalah bagaimana meskipun kita bermain di atas panggung dan dihadapan
        banyak orang, kita harus tetap rendah hati. Selain itu, dalam pelayanan
        saya juga dapat mengembangkan diri baik secara jasmani maupun rohani.
        
        \end{parts}
        \end{solution}

        \pagebreak
        \question
        Sejak dimulainya jaman modern di abad 16, terjadi perubahan paradigma
        manusia dalam memahami dunia kehidupannya. Dari berita di atas:
        \begin{parts}
            \part Jelaskan pergeseran paradigma Kosmosentris - Teosentris -
                  Antroposentris dan jelaskan apa itu sekularisme dan 
                  sekularisasi di zaman modern!
            \part Uraikanlah bagaimana manusia jatuh pada kehidupan yang 
                  materialistik di zaman modern! 
        \end{parts}

        \begin{solution}
        \begin{parts}
        
            \part Paham kosmosentris muncul pada zaman primitif. 
            Kosmosentris merupakan suatu paham bahwa alam sebagai pusat
            segalanya. Paham ini memahami bahwa semua kajian dan fokus pemikiran
            filsafat berpusat kepada alam. Oleh karena itu, manusia pada zaman 
            itu sangat menghormati alam. Karena pada dasarnya manusia adalah
            bagian dari alam itu sendiri dan manusia memiliki keyakinan bahwa
            Hyang Ilahi itu hadir dalam alam, sehingga relasi manusia dan alam
            pada zaman itu begitu harmonis.

            Lalu pada abad pertengahan, muncul paham Teosentris yang menjadi
            paham utama di masa itu. Teosentris menekankan kebenaran Allah dan
            Alkitab, serta menguatkan peran gereja pada kehidupan masyarakat 
            Eropa. Akibatnya, kebebasan berpendapat serta berpikir menjadi
            terbatas, dan ilmu pengetahuan menjadi tidak berkembang dan bahkan
            dicap sebagai sesuatu yang sesat oleh gereja. Paham ini membuat
            banyak filsafat serta pemikir merasa jenuh karena mereka ingin
            melepaskan diri dari doktrin agama dan gereja. Mereka berpendapat
            bahwa manusia harusnya mengembangkan akal budi yang telah diberikan
            oleh Tuhan.

            Hal ini menyebabkan lahirnya paham Anthroposentris. Paham ini 
            bertolak belakang dengan paham teosentris, dimana paham teosentris 
            menekankan Tuhan sebagai pusat, sedangkan anthroposentris menekankan
            manusia sebagai pusat. Anthroposentris sendiri berasal dari kata 
            \emph{Anthropos} yang berarti manusia, serta \emph{sentris} yang
            berarti pusat. Paham ini memahami bahwa dengan akal budi, manusia
            mampu menciptakan banyak hal yang berguna untuk manusia lain dan 
            memajukan masyarakat. Anthroposentris sendiri dimulai dari masa
            \emph{Renaissance}, masa dimana kebudayaan dan peradaban Eropa
            kembali lahir, sehingga banyak orang yang mulai mempelajari kembali
            filsafat dan seni Yunani dan Romawi kuno. Masa \emph{Renaissance}
            melahirkan banyak inovasi dan penemuan baru dalam ilmu pengetahuan
            serta berkembangnya filsafat di benua Eropa. Setelah \emph{Renaissance},
            Anthroposentris masuk ke dalam masa \emph{Aufklarung}; suatu kata
            dari bahasa Jerman yang berarti Pencerahan. Pada masa ini,
            orang-orang menjadi percaya diri terhadap kemampuannya untuk
            menciptakan kemajuan dan kebahagaiaan melalui ilmu pengetahuan yang
            ditemukan dan diciptakannya. Kita bisa melihat ilmuwan-ilmuwan yang
            berasal dari masa ini yaitu Isaac Newton, John Locke, serta Francis
            Bacon.

            Kemudian pada abad ke-18, muncul paham Sekular, dimana sekular
            sendiri berarti hal-hal duniawi yang bersifat tidak suci dan tidak
            ada kaitannya dengan agama. Sekular bisa dibagi menjadi dua bentuk,
            Sekularisasi yang merupakan proses perubahan seseorang mengikuti
            hal-hal yang berbau duniawi serta melepaskan nilai-nilai keagamaannya.
            Lalu ada sekularisme adalah suatu paham yang mengejar kesejahteraan
            duniawi dengan cara meninggalkan nilai serta norma agama. Contoh dari
            sekular ini bisa berupa pergaulan bebas yang semakin marak di
            kalangan pelajar, menggunakan ekonomi kapitalisme, serta kebebasan
            berpendapat.
        
            \part
            Kehidupan materialistik manusia yang ada di zaman modern ini muncul
            dari keadaan ekonomi dunia yang semakin baik sehingga banyak manusia
            dapat memenuhi kebutuhan materialnya. Selain itu, sifat materialistik
            ini juga didasari dari paham sekularisme yang mulai muncul di kalangan
            masyarakat menengah ke atas. Karena Paham sekularisme ini mengakibatkan
            munculnya dorongan perkembangan paham kapitalisme serta konsumerisme
            yang membuat manusia makin kehilangan nilai-nilai transendennya.
            Sehingga orang-orang akan lebih mengejar hal-hal yang bersifat sekular
            atau duniawi seperti kekuasaan, material, hingga seksual; ketimbang
            mengejar hal-hal yang bersifat transenden. Hal-hal sekular disini bisa
            berupa harta, kekayaan, atau bahkan pergaulan demi mendapatkan
            kebahagaiaan yang ia ingini.

        \end{parts}
        \end{solution}

        \pagebreak
        \question
        Kerusakan alam terjadi karena manusia memandang diri sebagai puncak
        evolusi dan merasa layak menggunakan alam sesuai dengan kehendaknya.
        Paradigma antroposentris dengan kemajuan sains dan teknologi seolah
        mendukung eksploitasi besar-besaran terhadap alam. Akibatnya alam
        rusak dan bencana pun bermunculan. Manusia yang konsumtif menambah
        cepat kerusakan alam sekitar:
        \begin{parts}
            \part Jelaskan 3 dampak antroposentrisme pada kerusakan alam!
            \part Berikan 5 contoh konsumerisme yang berdampak pada kerusakan
                  lingkungan hidup dan tanggapan Anda!
            \part Bagaiman pandangan agama Anda terhadap tanggung jawab menjaga
                  lingkungan? (sertakan dalil atau ayat jika ada)
        \end{parts}

        \begin{solution}
        \begin{parts}
        \part \begin{enumerate}
            \item \textbf{Pemanasan Global} \\
                Pemanasan Global merupakan suatu fenomena banyaknya gas
                panas yang diproduksi dari aktivitas manusia yang
                menyebabkan banyaknya juga panas yang terperankap dalam
                atmosfer untuk dipantulkan kembali ke bumi. Efek dari Pemanasan
                global ini kita dapat rasakan pada suhu permukaan bumi yang
                dari tahun ke tahun terasa semakin panas, sehingga menyebabkan
                naiknya permukaan laut karena melelehnya es yang ada di Antartika.

            \item \textbf{Polusi dan pencemaran} \\  
                Polusi merupakan salah satu hasil pembuangan dari
                aktivitas manusia seperti limbah pabrik, asap pembuangan
                kendaraan bermotor, dan juga sampah plastik.
                Hal ini mengakibatkan susahnya mendapatkan air dan udara
                yang bersih, serta hancurnya terumbu karang yang
                menyebabkan banyak biota laut mati.
            
            \item \textbf{Eksploitasi Kekayaan Alam} \\
                Eksploitasi yang dimaksud disini adalah suatu tindakan
                pengambilan sumber daya alam yang secara berlebihan hingga SDA
                tersebut berkurang. Penyebabnya adalah sifat manusia yang semakin
                rakus dan tidak lagi hormat terhadap alam sehingga menyebabkan
                rusaknya alam dan juga menyebabkan dua kejadian yang telah
                disebutkan diatas.
        \end{enumerate}
        
        \part \begin{enumerate}
            \item Setiap tahun membeli gawai baru \\
                Gawai merupakan sesuatu yang menjadi bagian penting dalam
                kehidupan manusia sekarang. Tapi jika kita terlalu sering
                mengganti gawai, tentunya sangat berdampak buruk pada lingkungan
                karena secara tidak langsung kita memproduksi sampah elektronik
                atau yang disebut sebagai \emph{e-waste}. Menurut data dari
                Statista, manusia memproduksi 53.6 juta metrik ton \emph{e-waste}
                pada tahun 2019 sendiri. Oleh karena itu, saran saya adalah
                jangan terlalu sering mengganti gawai. Pakailah gawai yang Anda
                masih miliki sampai benar-benar rusak atau memang darurat untuk
                diganti.

            \item Tergiur oleh diskon besar-besaran \\
                Dalam dunia yang serba digital ini, banyak toko-toko online yang
                mulai membuka dirinya. Dan bahkan biasanya mereka menawarkan diskon
                besar-besaran hampir setiap bulan untuk pengguna setianya. Dampaknya
                adalah kebanyakan orang akan membeli barang-barang yang seharusnya
                tidak diperlukan karena diskon tersebut. Sehingga akan bermunculan
                banyak sampah-sampah baru akibat dari hal tersebut. Oleh karena itu,
                kurangi membeli barang hanya karena diskonnya saja dan membeli
                jika memang sangat dibutuhkan.

            \item Terlalu sering menggunakan kantong plastik \\
                Biasanya ketika kita selesai berbelanja, kita akan diberikan
                kantong plastik untuk membawa semua belanjaan kita. Tapi kantong
                plastik sebenarnya memiliki efek buruk pada lingkungan, karena
                plastik merupakan salah satu material yang susah untuk diuraikan
                sehingga menyebabkan polusi pada tanah maupun air. Dari data
                \emph{TheWorldCounts.com}, dunia memproduksi 160.000 kantong
                plastik tiap detik, dan hanya kurang dari 1 persen saja yang
                didaur ulang. Lalu apa yang dapat kita lakukan untuk mengurangi
                hal tersebut? Kita dapat membeli kantong belanja yang ramah
                lingkungan ketika kita berbelanja. Karena dengan hal tersebut,
                kita dapat mengurangi penggunaan kantong plastik di dunia.

            \item Kebanyakan memesan \emph{fast food} \\
                \emph{Fast Food} atau masakan cepat saji adalah makanan yang sudah
                disiapkan dan bisa diajikan dalam waktu cepat (biasanya sekitar
                5-10 menit). Biasanya makanan \emph{fast food} ini selain tidak
                sehat untuk kesehatan manusia, tapi juga membawa dampak buruk
                bagi lingkungan. \emph{Fast food} memproduksi dua macam sampah,
                yaitu sampah kemasan serta sampah makanan. Dari data \emph{Sintelly},
                Restoran \emph{fast food} di dunia menggunakan 40\% untuk kemasannya,
                serta menyisakan 55\% makanan sisa. Oleh karena itu, kita harus
                mengurangi konsumsi makanan cepat saji, dan mulai mengonsumsi
                makanan-makanan sehat seperti sayuran, serta memasak sendiri makanan
                kita.

            \item Membeli kendaraan yang tidak ramah lingkungan \\
                Kendaraan bermotor memiliki sesuatu yang bernama emisi gas buang.
                Emisi ini biasanya mengandung zat-zat bahaya seperti Karbon
                Monoksida, Karbon Dioksida, serta Hidrokarbon yang sangat berbahaya
                jika dihirup oleh manusia, dan juga untuk lapisan ozon bumi. Oleh
                karena itu sebelum membeli kendaraan bermotor, kita harus meninjau
                ulang emisi gas buang kendaraan tersebut, atau menggunakan kendaraan
                listrik untuk mengurangi dampak buruk yang dihasilkan oleh kendaraan
                bermotor terhadap lingkungan.

        \end{enumerate}

        \part 
        % TODO Disini nanti bahas tentang Kejadian 1 tentang Allah yang memberikan wewenang
        % kekuasaan merawat bumi kepada manusia. (tambahin sedikit)
        % Manusia sudah pernah kena murka Allah pada Yeremia 2:7 dimana Allah memberikan
        % Israel tanah yang baik tapi ga dijaga. (Panjangin)

        Dalam Alkitab, manusia telah diberikan wewenang dalam mengurus alam. 
        Kita dapat melihatnya pada Kejadian 1:28 dalam terjemahan BIMK:
        \begin{displayquote}
            \emph{Kemudian diberkati-Nya mereka dengan ucapan “Beranakcuculah
            yang banyak, supaya keturunanmu mendiami seluruh muka bumi serta
            menguasainya. \textbf{Kamu Kutugaskan mengurus ikan-ikan, burung-burung, dan
            semua binatang lain yang liar.}}
        \end{displayquote}
        Dalam terjemahan ini, Allah telah memberikan manusia tugas untuk mengurus
        ikan-ikan, burung-burung, serta binatang lainnya, atau bisa dipahami juga
        dengan mengurus alam ini. 

        Selain itu, manusia sebenarnya sudah pernah mendapatkan murka dari Allah
        karena tidak menjaga lingkungannya. Hal ini kita dapat melihatnya dalam
        Yeremia 2:7 dalam Terjemahan Baru:
        \begin{displayquote}
            \emph{Ke negeri yang subur Kuantar mereka \\
            untuk menikmati hasil-hasil dan segala yang baik di sana. \\
            \textbf{Tapi di situ tanah-Ku mereka najiskan \\
            dan milik-Ku mereka jadikan sesuatu yang menjijikkan.}}
        \end{displayquote} 

        Dari ayat-ayat tersebut kita dapat mempelajari bahwa Allah sendiri telah
        memberikan kita wewenang dan tugas untuk menjaga ciptaannya dan sekarang
        adalah tugas kita untuk melaksanakan wewenang tersebut dengan baik.
        \end{parts}
        \end{solution}
    \end{questions}

    \pagebreak

    \printbibliography
\end{document}

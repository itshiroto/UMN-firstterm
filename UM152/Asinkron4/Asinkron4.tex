\documentclass[a4paper,11pt,answers]{exam}
\usepackage[margin=2cm]{geometry}
\usepackage{plex-otf}
\usepackage[bahasai]{babel}

\setlength{\parskip}{1em} 
\setlength{\parindent}{0pt}
\linespread{1.3}

\renewcommand{\solutiontitle}{
    \noindent\textbf{Jawab:}\par\noindent
}

\newcommand{\mytitle}{Religion - Asinkron Week 13}
\newcommand{\theauthor}{Rivo Juicer Wowor}
\newcommand{\affiliation}{00000059635}

\lhead{\footnotesize{\textbf{\mytitle}}}%
\rfoot{\thepage}%
\cfoot{}
\lfoot{\footnotesize{\theauthor \hspace{1pt} (\affiliation)}}
\pagestyle{headandfoot}
 
\begin{document}
    \begin{questions}
        \question \emph{Konflik antarumat beragama diantaranya disebabkan oleh sikap 
        beragama yang kurang dewasa dan juga ketidakmampuan merespon situasi dan
        kondisi zaman yang plural dan tanpa batas.} \par
        Apa yang dimaksud dengan sikap beragama yang kurang dewasa itu menurut anda dan berikan penjelasan?

        \begin{solution}
            Menurut saya, sikap beragama yang kurang dewasa itu bisa seperti sikap intoleransi, keras kepala
            dan mengekslusikan agama lain. Karena setiap agama memiliki tujuan untuk memanusiakan manusia dan
            mendorong masyarakat untuk memiliki semangat persatuan serta solidaritas. Dan orang yang keras kepala
            disertai dengan intoleran ini sama saja seperti merendahkan manusia lain dan juga tidak menghargai 
            solidaritas sehingga berlawanan dengan tujuan agama yang telah disebutkan tadi. Hal-hal ini yang
            menyebabkan seseorang dapat dikatakan memiliki sikap beragama yang kurang dewasa.
        \end{solution}
 
        \question \emph{Agama diakui sebagai kekuatan pembebas bagi manusia.} \par
        Apa yang dimaksud dengan Agama diakui sebagai kekuatan pembebas bagi manusia itu menurut anda dan berikan penjelasan?

        \begin{solution}
            Menurut saya, agama bisa dikatakan sebagai kekuatan pembebas bagi manusia karena agama menyentuh alam rohani seorang manusia. Sehingga ketika seseorang mengalami masalah batin atau dalam diri sendiri, kemungkinan besar agama dapat menjadi solusi atau pemandu untuk menyelesaikan masalah orang tersebut.
        \end{solution}
 
        \question Berilah contoh bidang-bidang kerjasama antar umat beragama yang selama ini pernah anda lakukan baik secara pribadi maupun dengan melibatkan/bersama dengan lembaga keagamaan anda! Jikalau anda belum pernah melakukannya, apa penyebabnya dan bisakah anda memulainya?

        \begin{solution}
            Yang pernah saya alami adalah dalam menjaga parkir bersama komunitas pemuda gereja saya ketika hari Idul Fitri di salah satu masjid di kota saya. Jadi biasanya pemuda-pemudi antar-agama yang ada di kota saya bertugas untuk menjaga parkir ketika hari besar suatu agama dirayakan. Yang Kristen menjaga parkir di masjid ketika Idul Fitri, yang Muslim menjaga parkir di gereja ketika Hari Raya Natal, dan seterusnya. Sehingga biasanya hubungan antar umat beragama bisa lebih dekat dan harmonis lagi. Dan hal tersebut saya bisa aplikasikan ketika saya bersekolah di SMA saat acara keagamaan besar.
        \end{solution}

    \end{questions}
\end{document}
\PassOptionsToPackage{unicode}{hyperref}
\PassOptionsToPackage{hyphens}{url}
%
\documentclass[
  11pt,
  answers  
]{exam}
\usepackage[fleqn]{amsmath}
\usepackage[]{plex-otf}
\usepackage{iftex}
\usepackage[a4paper, margin=2.5cm]{geometry}
\usepackage{graphicx}
\usepackage[sorting=none, backend=biber]{biblatex}
% \ifPDFTeX
%   \usepackage[T1]{fontenc}
%   \usepackage[utf8]{inputenc}
%   \usepackage{textcomp} % provide euro and other symbols
% \else % if luatex or xetex
%   \usepackage{unicode-math}
%   \defaultfontfeatures{Scale=MatchLowercase}
%   \defaultfontfeatures[\rmfamily]{Ligatures=TeX,Scale=1}
% \fi
% Use upquote if available, for straight quotes in verbatim environments
\IfFileExists{upquote.sty}{\usepackage{upquote}}{}
\IfFileExists{microtype.sty}{% use microtype if available
  \usepackage[]{microtype}
  \UseMicrotypeSet[protrusion]{basicmath} % disable protrusion for tt fonts
}{}
\makeatletter
\makeatother
\usepackage{xcolor}
\IfFileExists{xurl.sty}{\usepackage{xurl}}{} % add URL line breaks if available
\IfFileExists{bookmark.sty}{\usepackage{bookmark}}{\usepackage{hyperref}}
\urlstyle{same} % disable monospaced font for URLs
\setlength{\emergencystretch}{3em} % prevent overfull lines
\providecommand{\tightlist}{%
  \setlength{\itemsep}{0pt}\setlength{\parskip}{0pt}}
\setcounter{secnumdepth}{-\maxdimen} % remove section numbering
\newlength{\cslhangindent}
\setlength{\cslhangindent}{1.5em}
\newlength{\csllabelwidth}
\setlength{\csllabelwidth}{3em}
\newlength{\cslentryspacingunit} % times entry-spacing
\setlength{\cslentryspacingunit}{\parskip}
\newenvironment{CSLReferences}[2] % #1 hanging-ident, #2 entry spacing
 {% don't indent paragraphs
  \setlength{\parindent}{0pt}
  % turn on hanging indent if param 1 is 1
  \ifodd #1
  \let\oldpar\par
  \def\par{\hangindent=\cslhangindent\oldpar}
  \fi
  % set entry spacing
  \setlength{\parskip}{#2\cslentryspacingunit}
 }%
 {}
\usepackage{calc}
\newcommand{\CSLBlock}[1]{#1\hfill\break}
\newcommand{\CSLLeftMargin}[1]{\parbox[t]{\csllabelwidth}{#1}}
\newcommand{\CSLRightInline}[1]{\parbox[t]{\linewidth - \csllabelwidth}{#1}\break}
\newcommand{\CSLIndent}[1]{\hspace{\cslhangindent}#1}
\ifLuaTeX
  \usepackage{selnolig}  % disable illegal ligatures
\fi
\renewcommand{\solutiontitle}{\noindent\textbf{Jawab:}\par\noindent}
\newcommand{\mytitle}{Religion (UM 152)}
\newcommand{\theauthor}{Rivo Juicer Wowor}
\newcommand{\affiliation}{00000059635}

% \title{\textbf{\mytitle}}
% \author{\theauthor \
		% \small{\affiliation}}
% \date{}

% \usepackage{fancyhdr}
\lhead{\footnotesize{\textbf{\mytitle}}}%
\rfoot{\thepage}%
\cfoot{}
\lfoot{\footnotesize{\theauthor \hspace{1pt} (\affiliation)}}
\pagestyle{headandfoot}

% \pagestyle{plain}
\setlength{\parindent}{2em}
\renewcommand{\baselinestretch}{1.5}
\hbadness=99999
% \addbibresource{../../ref/library.bib}
\usepackage{enumitem}
\unframedsolutions

\begin{document}
	\begin{titlepage}
		\centering
		\vspace{2cm}
		\includegraphics[width=0.5\textwidth]{../../ref/logoUMN.png}\par\vspace{1cm}
		\vspace{1.5cm}
		\Large{Midterm Assignment} \par
		\vspace{1cm}
		\huge{\textbf{\mytitle}} \par
		\vspace{1.5cm}
		\Large{\theauthor} \par
		\emph{\affiliation} \par
		\vfill
		\today
		\end{titlepage}	

    \begin{questions}
      \question
      Sebagai mahluk, manusia terdiri dari tiga dimensi, yaitu bertubuh, berjiwa, dan berrohani.
      Sebagaimana yang diajarkan oleh agama yang kita anut, maka kita perlu mengembangkannya seturut
      dengan hakikat manusia yang diberikan oleh pencipta. Bagaimana Anda mengembangkan hakikat yang anda
      di masa pandemi seperti sekarang ini? Berikan contoh konkret yang telah Anda lakukan selama ini.
      \begin{solution}
        Setiap manusia memiliki tiga dimensi, yaitu dimensi tubuh, dimensi psikis, dan dimensi spiritual.
        Masing-masing memiliki fungsinya sendiri. Yang pertama dimensi tubuh yang merupakan dimensi dasar
        untuk membentuk personalitas atau kedirian manusia, seperti gaya hidup, kesehatan, dan lainnya.
        Kemudian ada dimensi psikis yang merupakan dimensi dasar untuk membentuk pola pikir seorang manusia,
        seperti berintuisi, berencana, dan lainnya. Lalu yang terakhir ada dimensi spiritual yang merupakan
        dimensi kecenderungan manusia terhadap keabadian, seperti kepercayaan, dan lainnya. Ketiga dimensi ini
        tidak dapat dipisahkan karena ketiganya merupakan suatu hakikat dan dasar dari manusia.

        Selama pandemi ini, untuk mengembangkan diri saja merupakan suatu yang sedikit sulit. Apalagi banyak orang
        yang mengalami depresi atau kerusakan pada dimensi psikis mereka. Tapi ada beberapa cara yang saya lakukan
        selama pandemi untuk mengembangkan ketiga dimensi saya untuk menjadi lebih baik. Untuk dimensi tubuh sendiri
        adalah dengan beristirahat yang cukup, rajin makan dan minum yang bergizi, dan juga mengontrol waktu pekerjaan
        saya agar tidak berlebih. Untuk dimensi psikis, saya selalu menantang diri saya untuk mengerjakan masalah-masalah
        \emph{programming} sekaligus belajar mengenai hal-hal yang baru dalam dunia teknologi. Terakhir adalah dimensi
        spiritual dimana saya selalu rajin berdoa kepada Tuhan, dan melayani dalam pelayanan \emph{Broadcasting} di gereja
        saya.
      \end{solution}

      \pagebreak

      \question
      Kebutuhan mendasar merupakan kebutuhan yang tanpanya manusia tidak dapat hidup. Di masa kini, terdapat
      berbagai kesulitan dan juga kekangan yang menghambat kebebasan kita. Bagaimana kemudian anda menyikapi
      hal tersebut melalui konsep yang telah anda pelajari.

      \textbf{Soal: }
      
      Berkaitan dengan situasi pandemi COVID-19 ini, kita dapat menganalisis dari pandangan Abraham Maslow
      dan Viktor Frankl.
      \begin{parts}
        \part Abraham Maslow mengemukakan teorinya tentang hierarki kebutuhan dasar manusia. Menurut Anda dalam situasi
        pandemi ini, kebutuhan dasar mana yang sangat dibutuhkan oleh manusia? Beri alasan Anda!
        \part Viktor Frankl berpandangan bahwa manusia mampu memaknai hidupnya dalam keadaan apapun, termasuk dalam situasi
        sulit atau penderitaan. Situasi pandemi berdampak pada penderitaan manusia dan membuat sebagian orang kehilangan
        makna hidup. Bagi Anda pribadi, bagaimana cara Anda menemukan makna hidup dalam situasi pandemi sekarang ini? 
        Bagaimana Anda menghubungkan pengalaman? Uraikan!
      \end{parts}
      \begin{solution}
        \begin{parts}
          \part
          Menurut saya, Manusia sekarang sangat memerlukan \emph{Social Needs} atau kebutuhan sosial jika berdasarkan
          Teori hierarki kebutuhan dasar manusia dari Abraham Maslow. Karena akhir-akhir ini, banyak orang yang kehilangan
          kasih sayang dari orang-orang terdekatnya dan kehilangan pekerjaan akibat pandemi ini. Dampaknya, tak sedikit
          dari orang-orang tersebut merasa depresi dan memiliki gangguan psikologis. Oleh karena itu, saya merasa diperlukannya
          afeksi dan kasih sayang bagi banyak orang untuk memenuhi salah satu dari lima kebutuhan manusia ini.
          
          \part
          Menurut saya, salah satu penyebab orang kehilangan makna hidup adalah karena bosan, dan salah satu cara saya dalam
          menemukan makna hidup adalah dengan menemukan dan mempelajari hal-hal yang baru. Cara tersebut saya lakukan untuk
          merangsang otak saya agar tetap penasaran dan bekerja. Seperti pada tahun ini, saya mencoba untuk mempelajari
          \emph{Web Development} selama setahun ini dan banyak hal-hal baru yang membuat saya tertarik dan terus belajar. Dan
          hasilnya adalah hidup saya memiliki makna salah satunya karena dengan belajar tersebut.
        \end{parts}
      \end{solution}

      \pagebreak

      \question
      Jelaskan mengapa manusia membutuhkan orang lain untuk kelangsungan hidupnya! Beri penjelasan sesuai dengan teori-teori
      sosial dan budaya yang sudah dipelajari.
      \begin{solution}
        Sudah banyak ilmuwan yang mempelajari mengenai manusia sebagai sebuah makhluk sosial. Salah satu teori yang terkenal
        adalah \emph{Zoon Politicon} dari Aristotles yang memiliki pengertian Manusia pada dasarnya hidup bersama-sama dan
        selalu berinteraksi satu dengan yang lain. Oleh karena itu manusia memiliki kecenderungan untuk hidup dalam suatu
        komunitas atau masyarakat. Dan dari kehidupan komunitas tersebut, timbullah suatu budaya, karena kebudayaan itu muncul,
        hidup dan berkembang karena adanya suatu komunitas (masyarakat). Seperti yang dikatakan Ki Hajar Dewantara bahwa 
        \emph{"Kebudayaan adalah buah budi manusia dalam hidup bermasyarakat"}. Kita bisa melihat contohnya dalam suatu sistem religi
        atau agama di dunia, sistem bahasa antar masyarakat, dan juga kebudayaan yang ada. Pada akhirnya, faktor masyarakat dan budaya inilah
        yang menekankan dan menguatkan pernyataan manusia membutuhkan orang lain untuk kelangsungan hidupnya.
      \end{solution}

      \pagebreak

      \question
      Bagaimana suara hati berperan saat anda menghadapi permasalahan yang dilematis di dalam kehidupan anda? (Sebutkan dan 
      uraikan permasalahan anda, lalu analisa dengan menggunakan teori yang telah dipelajari)
      \begin{solution}
        Terkadang dalam hidup, kita menghadapi permasalahan yang dilematis. Dimana kita harus memilih ataupun memutuskan satu dari
        dua atau tiga pilihan yang berbeda. Dan memilih pilihan tersebut tidaklah mudah. Salah satu contohnya adalah dalam memilih jurusan
        untuk melanjutkan ke pendidikan tinggi. Pengalaman ini tentu saja pernah dirasakan oleh hampir semua siswa SMA kelas tiga ketika mendekati
        kelulusan. Tentunya dalam memilih jurusan, selalu didasari oleh berbagai alasan seperti \emph{"Tidak direstui orang tua"}, \emph{"Ikut teman",}
        dan lain sebagainya. Lalu bagaimana cara kita dalam menghadapi permasalahan tersebut?

        Dalam setiap diri manusia, ada suatu proses kesadaran yaitu \emph{Suara Hati}. Suara hati ini merupakan proses kesadaran
        akan kewajiban dan tanggungjawab seseorang sebagai manusia dalam situasi nyata. Penjelasan lengkapnya kita dapat lihat
        dalam teori Prof. Dr. Magnis Suseno, SJ yang mengemukakan bahwa suara hati merupakan kesadaran dalam batin seseorang bahwa
        Ia berkewajiban mutlak untuk selalu menghendaki apa yang menjadi kewajiban dan tanggung jawab dia. Dan Suara Hati ini dipengaruhi
        oleh teori dari Sigmund Freud mengenai tiga unsur diri manusia yang dinamis, yaitu \emph{Id} (aku yang tidak sadar); kepribadian
        asli bawaan setiap manusia sejak lahir, \emph{Ego} (aku yang sadar); pengenalan akan realitas, rasionalitas tetapi tetap berpusat
        pada diri sendiri, dan \emph{Super-Ego} (aku yang ideal); perkembangan dari ego dimana ada proses kesadaran tentang nilai-nilai
        moral untuk mencapai kualitas hidup yang lebih tinggi. Setelah itu, ada kurang lebih tiga sumber menurut \emph{Prof. Dr. Magnis 
        Suseno} yang disebut sebagai lembaga normatif dalam mengasah suara hati. Ketiga lembaga tersebut adalah masyarakat, \emph{super-ego},
        dan ideologi. Tapi terkadang ketiga lembaga ini kurang meyakinkan untuk bisa diikuti sehingga kita dapat memutuskannya melalui suara hati
        kita sendiri.

        Dari penjelasan tersebut, kita dapat memakai lembaga normatif dan suara hati dalam menghadapi pengalaman dilematis. Kita ambil contoh
        sebelumnya yaitu ketika saya memilih jurusan pendidikan tinggi. Biasanya saya akan mereferensi terhadap tiga lembaga normatif tadi dalam memilih 
        jurusan yang saya akan pilih nantinya. Awalnya saya ingin melanjutkan pendidikan tinggi dalam bidang seni musik. Tentunya saya akan berpikir mengenai
        cita-cita saya, kondisi orang tua, dan juga ajaran-ajaran dari keluarga saya. Tapi ketiga lembaga normatif ini kurang meyakinkan saya dalam mengambil
        keputusan untuk melanjutkan pendidikan dalam musik. Cita-cita saya memanglah menjadi seorang \emph{audio engineer} dan musisi, tapi orang tua saya
        tidak merestui pilihan saya tersebut karena kondisi keuangan keluarga saya yang masih dibilang cukup. Oleh karena itu, saya mengambil keputusan akhir
        dari suara hati saya yang akhirnya menimbang untuk mengejar cita-cita kedua saya, yaitu menjadi seorang \emph{software engineer} dan memilih jurusan
        Teknik Informatika atau \emph{Computer Science} sebagai jurusan saya untuk melanjutkan pendidikan tinggi.

      \end{solution}
    \end{questions}
    \pagebreak
    % \printbibliography

		
\end{document}

% \PassOptionsToPackage{unicode}{hyperref}
\PassOptionsToPackage{hyphens}{url}
%
\documentclass[
  12pt,
]{article}
% \usepackage[fleqn]{amsmath}
\usepackage[]{mathptmx}
\usepackage{iftex}
\usepackage[a4paper]{geometry}
\usepackage{graphicx}
% \usepackage[sorting=none, backend=biber]{biblatex}
% \ifPDFTeX
%   \usepackage[T1]{fontenc}
%   \usepackage[utf8]{inputenc}
%   \usepackage{textcomp} % provide euro and other symbols
% \else % if luatex or xetex
%   \usepackage{unicode-math}
%   \defaultfontfeatures{Scale=MatchLowercase}
%   \defaultfontfeatures[\rmfamily]{Ligatures=TeX,Scale=1}
% \fi
% Use upquote if available, for straight quotes in verbatim environments
\IfFileExists{upquote.sty}{\usepackage{upquote}}{}
\IfFileExists{microtype.sty}{% use microtype if available
  \usepackage[]{microtype}
  \UseMicrotypeSet[protrusion]{basicmath} % disable protrusion for tt fonts
}{}
\makeatletter
\makeatother
\usepackage{xcolor}
\IfFileExists{xurl.sty}{\usepackage{xurl}}{} % add URL line breaks if available
\IfFileExists{bookmark.sty}{\usepackage{bookmark}}{\usepackage{hyperref}}
\urlstyle{same} % disable monospaced font for URLs
\setlength{\emergencystretch}{3em} % prevent overfull lines
\providecommand{\tightlist}{%
  \setlength{\itemsep}{0pt}\setlength{\parskip}{0pt}}
% \setcounter{secnumdepth}{-\maxdimen} % remove section numbering
\newlength{\cslhangindent}
\setlength{\cslhangindent}{1.5em}
\newlength{\csllabelwidth}
\setlength{\csllabelwidth}{3em}
\newlength{\cslentryspacingunit} % times entry-spacing
\setlength{\cslentryspacingunit}{\parskip}
\newenvironment{CSLReferences}[2] % #1 hanging-ident, #2 entry spacing
 {% don't indent paragraphs
  \setlength{\parindent}{0pt}
  % turn on hanging indent if param 1 is 1
  \ifodd #1
  \let\oldpar\par
  \def\par{\hangindent=\cslhangindent\oldpar}
  \fi
  % set entry spacing
  \setlength{\parskip}{#2\cslentryspacingunit}
 }%
 {}
\usepackage{calc}
\newcommand{\CSLBlock}[1]{#1\hfill\break}
\newcommand{\CSLLeftMargin}[1]{\parbox[t]{\csllabelwidth}{#1}}
\newcommand{\CSLRightInline}[1]{\parbox[t]{\linewidth - \csllabelwidth}{#1}\break}
\newcommand{\CSLIndent}[1]{\hspace{\cslhangindent}#1}
\ifLuaTeX
  \usepackage{selnolig}  % disable illegal ligatures
\fi
% \renewcommand{\solutiontitle}{\noindent\textbf{Jawab:}\par\noindent}
\newcommand{\mytitle}{Religion (UM 152)}
\newcommand{\theauthor}{Rivo Juicer Wowor}
\newcommand{\affiliation}{00000059635}

% \title{\textbf{\mytitle}}
% \author{\theauthor \
		% \small{\affiliation}}
% \date{}

% \usepackage{fancyhdr}
% \lhead{\footnotesize{\textbf{\mytitle}}}%
% \rfoot{\thepage}%
% \cfoot{}
% \lfoot{\footnotesize{\theauthor \hspace{1pt} (\affiliation)}}
% \pagestyle{headandfoot}

% \pagestyle{plain}
\setlength{\parindent}{2em}
\renewcommand{\baselinestretch}{1.5}
\hbadness=99999
% \addbibresource{../../ref/library.bib}
\usepackage{enumitem}
% \unframedsolutions

\begin{document}
	\begin{titlepage}
		\centering
    \vspace{20cm}
		\huge{\textbf{Sekularisme dan Formalisme Agama \\
    sebagai Tantangan dalam Beragama}} \par
		\vspace{2cm}
		\Large{Makalah Religiusitas} \par
		\vspace{2cm}
		\includegraphics[width=0.7\textwidth]{../../ref/logoUMN.png}\par\vspace{1cm}
		\vspace{1.5cm}
		\Large{Kelompok 6} \par
    \begin{center}
		\large\emph{
      \begin{tabular}{c c}
        Rivo Juicer Wowor & Arfigo Ezra Pratama \\   
        Andrea Zoe Putri Sukonco & Muhammad Rajja Farabi  
      \end{tabular} \\ 
      \vspace{12pt}
      Yosi Sly Afrilia N}
    \end{center}\par
		\vfill
		\end{titlepage}	
    \newgeometry{margin=2cm}

    \pagebreak
    \section{Pendahuluan}
    \subsection{Latar Belakang}
    Munculnya sekularisme terjadi ketika adanya trauma pada bangsa Eropa terhadap kaum gereja yang ingin berkuasa di setiap permasalahan. Sekularisme tersebut berdampak pada pendidikan, aspek sosial, politik, bahkan agama juga (terlihat pada agama mayoritas dan minoritas), sehingga mereganglah relasi antar agama yang kemudian membuat agama minoritas dalam posisi sulit. Disamping sekularisme, terdapat juga paham formalisme yang saling terhubung. Dengan adanya teknologi, orang-orang jauh lebih mudah untuk mendapatkan informasi. Sekularisme membuat orang menyatukan nilai-nilai agama dengan masalah dunia, dan formalisme membuat penampilan seseorang tersebut seperti ahli ilmu agama. Sehingga akan membuat pemikiran orang-orang langsung mempercayainya tanpa mencari bukti informasi yang pasti lebih benar, seperti melihat langsung pada Alkitab.
    \subsection{Rumusan Masalah}
    \begin{itemize}
      \item Apa itu sekularisme dan formalisme?
      \item Apa dampak dari sekularisme dan formalism dalam agama?
      \item Apa solusi untuk mengatasi sekularisme dan formalism dalam agama?

    \end{itemize}
    \pagebreak

    \section{Bahasan}
    \subsection{Sekularisme}
    Menurut salah satu ahli yaitu, George Jacob Holyoake, Sekularisme merupakan suatu sistem etik yang berdasar pada prinsip moral yang terlepas dari agama dan supranaturalisme. Maka secara umum, dapar disimpulkan bahwa konsep sekularisme ini adalah suatu konsep yang menyatakan bahwa paham agama tidak boleh dikaitkan dengan urusan-urusan lain diluar agama itu sendiri. Sekularisme juga bisa diartikan sebagai suatu pergerakan dalam masyarakat untuk menjauhi kepercayaan-kepercayaan gaib dan lebih mengarah ke kehidupan yang ada di bumi. Sekularisme memiliki ciri yang meyakini bahwa nilai keagamaan haruslah dibedakan dari nilai-nilai kehidupan dunia dan seluruh aspeknya, maka dari itu di dalam paham sekulerisme ini, paham agama tidak bisa dikaitkan dengan urusan lain selain agama itu sendiri.
      
    Namun tentu seperti banyak paham di luar sana, sekularisme ini memiliki kelebihan dan kekurangannya masing-masing. Sekularisme memiliki berbagai macam dampak positif, karena paham ini mewujudkan kebebasan dalam beragama. Maksudnya, kita bebas untuk memilih agama apa yang mau kita anut bahkan tidak menganut agama sekalipun. Lalu paham ini juga mewujudkan suatu keamanan dalam beragama. Karena paham ini lepas dari urusan-urusan negara dan politik, maka paham ini dapat mengurangi penyalahgunaan agama sebagai alat kontrol masyarakat. 
      
    Kekurangan yang dapat dirasakan dari paham sekularisme ini adalah seperti rusaknya moral agama dikalangan umat yang menggunakan paham ini. Karena masih banyak dari mereka yang berpikir bahwa agama hanya sebatas ibadah saja, sehingga dapat menghambat pertumbuhan kerohanian seseorang. Lalu juga umat-umat yang mengutamakan hal-hal yang bersifat materi. Karena menurut kamus \emph{Oxford}, sekularisme ini lebih bersifat ke hal-hal yang berbau duniawi atau materialisme, bukan berat kepada keagamaan maupun kerohanian. lalu, paham ini dapat memunculkan ide ide yang menyesatkan. Karena di dalam paham ini kita dipaksa untuk berpikir secara logika, namun logika seorang manusia belum tentu bisa sepenuhnya benar dan bijak, maka dari itu bisa saja muncul ide-ide yang menyesatkan, contohnya seperti radikalisme.  Selain itu sekularisme juga bisa berdampak buruk bagi hubungan agama mayoritas dan minoritas, dimana sekularisme ini menjadi pemicu tegangnya relasi antar agama yang membuat agama minoritas dalam posisi sulit.
      
    Menurut Prof Al-Attas, sekularisme setidaknya memiliki 3 ciri : 
    \begin{enumerate}
      \item \emph{Disenchantment of nature}, melihat alam sebagai objek yang dapat digunakan sepuasnya dan tidak memiliki keterkaitan dengan Tuhan. 
      \item \emph{Desacralization of politics}, ditiadakannya kesucian agama dari ranah politik (Agama dibatasi).
      \item \emph{Deconsecration of value}, agama merupakan urusan pribadi sehingga kemutlakannya dihapuskan dari kehidupan.
    \end{enumerate}

    \subsection{Formalisme}
    Formalisme agama merupakan penghayatan hidup beragama yang tertutup, eksklusif, tidak kritis, dan tanpa makna. Dan menurut sudut pandang islam bahwa Formalisme Islam yaitu institusionalisasi doktrin, simbol dan idiom keagamaan.
      
    Formalisme hanya menjadikan agama sebagai formalitas dan legalis. Orang yang memiliki sifat formalisme ini, cenderung lebih memperlihatkan kesucian dirinya terhadap  Tuhannya namun tidak memiliki ilmu agama yang pasti (samar-samar). Hanya mengikuti aturannya saja, namun tidak mengetahui ilmu sebenarnya. 
      
    Terdapat tiga faktor yang menurut kami menyebabkan terjadinya formalisme. Yang pertama yaitu kurangnya pemahaman dengan ilmu agama, akhirnya terjadi kekeliruan dan kesalahpahaman. Kemudian adalah Karena kurangnya pemahaman, bisa terjadinya ancaman perbedaan. Dan kebebasan agama saja belum bisa menjamin keselamatan. Lalu terakhir adalah Ilmu yang kurang bisa menyebabkan terjadinya kesesatan. Karena belum sepenuhnya tahu mana yang benar dan mana yang salah.
      
    Dan dampak-dampak yang dapat diakibatkan oleh formalisme adalah:
    \begin{enumerate}
      \item Selalu membawa agama untuk dipermasalahkan. \\
      Dikarenakan menganggap agama hal yang benar namun karena kurangnya pemahaman, sehingga informasi yang disangkut pautkan oleh agama bisa terjadi kesesatan 

      \item “Selalu meminta, namun tidak usaha” \\
      Kata-kata tersebut kami buat karena mencerminkan kepada seseorang yang termasuk formalisme. Hanya menginginkan keinginannya terwujud dengan cepat, namun tidak dilakukan dengan usaha.

      \item “Di samping ketekunan iman, hilangnya nilai kemanusiaan”. \\
      Kata-kata tersebut juga kami yang membuat.  Karena seseorang tersebut hanya memikirkan dirinya sendiri dan merasa dirinya paling benar, sehingga bisa terjadinya ancaman kepada orang lain jika tidak sepaham dengannya.

      \item “Agnostic berkedok agama”. \\
      Bisa disebut seperti itu karena formalisme kurangnya pemahaman dan hanya menjalankan kewajibannya saja, tidak tahu apa arti dan nilai dari kewajiban beribadahnya tersebut.
    \end{enumerate}
      
    \pagebreak

    \section{Penutup}

    \subsection{Kesimpulan}
    Dapat disimpulkan bahwa sekularisme adalah suatu konsep yang menyatakan bahwa paham agama tidak boleh dikaitkan dengan urusan-urusan lain diluar agama itu sendiri. Sedangkan Formalisme agama merupakan penghayatan hidup beragama yang tertutup, eksklusif, tidak kritis, dan tanpa makna. Dari kedua paham tersebut, dapat memicu peregangan antar hubungan mayoritas dan minoritas, membawa agama sebagai permasalahan sehari-hari, dan hilangnya nilai kemanusiaan.

    \subsection{Saran}
    Ada beberapa cara yang kita dapat lakukan untuk paham sekularisme dan formalisme:
    \begin{enumerate}
    \item \textbf{Kembali menjiwai iman rohani kita secara murni dan benar} \\
    Karena dengan kita kembali menjiwai iman rohani kita, maka kita akan bisa memilah hal2 yang buruk bagi kerohanian kita dan memilih keputusan yang bijak dan benar agar kita bisa menghindari diri kita dari ajaran agama sesat. Selain itu dengan menjiwai iman rohani kita, sifat kerohanian kita juga bisa berkembang seiring waktu.
      
    \item \textbf{Kritis terhadap kemajuan teknologi dan ilmu pengetahuan} \\
    Kemajuan iptek sangat berdampak bagi kehidupan sehari2 kita, maka dari itu perkembangan ini sangat penting untuk kita kembangkan dan kendalikan dengan baik dan benar, agar perkembangan teknologi ini dapat mengembangkan kehidupan beragama kita dan juga agar pemanfaatan teknologi ini tidak berujung kepada hal2 yang negatif seperti penyebaran hoax yang tidak benar, penistaan agama lewat media sosial, dan masih banyak lagi.
      
    \item \textbf{Berpikiran terbuka ketika mempelajari atau melihat suatu ajaran dan budaya} \\
    Seiring berkembangnya globalisasi, banyak sekali ajaran ajaran dan budaya budaya baru yang muncul. Maka dari itu, kita harus menyikapi hal tersebut dengan bijak agar kita tidak terjerumus kepada suatu ajaran atau budaya yang menyesatkan.
      
    \item \textbf{Menceritakan kembali pengalaman religius yang transenden kepada orang-orang terdekat} \\
    Memberikan kesaksian hidup yang pernah dialami dan terjadi secara nyata sebagai pengungkapan rasa syukur sekaligus memotivasi orang orang terdekat, dengan begitu kita dapat membantu mereka  membangun iman kepercayaan dan kerohanian.
      
    \item \textbf{Diperlukannya penafsiran ulang/pemahaman kembali terhadap simbol-simbol keagamaan} \\
    Dengan melakukan penafsiran ulang atau pemahaman kembali terhadap simbol simbol keagamaan, maka kita bisa memahami agama kita lebih dalam lagi, dan dengan melakukan hal tersebut, kita juga bisa membangun kerohanian kita lebih lagi. Hal tersebut dapat kita wujudkan bukan hanya dengan ibadah saja, namun bisa kita wujudkan dengan membaca kitab suci, melakukan saat teduh, menghargai orang dengan agama lain, dan berbuat baik kepada orang lain seperti membantu mereka yang membutuhkan. Karena agama bukanlah hanya sebuah gelar, namun perlu kita hayati dan dalami.
      
    \item \textbf{Kembali menjalankan agama berdasarkan pengalaman rohani yang terjadi pada diri kita, bukan ajaran-ajaran yang menyesatkan} \\
    Dengan menjalankan agama berdasarkan pengalaman rohani yang terjadi pada diri kita, maka kita bisa mengetahui mana ajaran yang baik yang dapat mengembangkan dan meningkatkan kerohanian kita, dan dengan ini kita juga bisa terhindar dari berbagai macam ajaran yang tidak sesuai dengan agama kita dan juga ajaran yang berdampak buruk bagi kehidupan sosial dan beragama seperti ajaran fanatik dan radikalisme.
    \end{enumerate}

    \pagebreak
    \section*{Daftar Pustaka}
    “Pengertian Sekularisme, Ciri, Bentuk, Dampak, dan Contohnya | DosenSosiologi.Com,” October 17, 2021. https://dosensosiologi.com/pengertian-sekularisme/, https://dosensosiologi.com/pengertian-sekularisme/.

    Hukamnas.com. “3 Dampak Sekularisme Terhadap Kehidupan Masyarakat,” March 2, 2019. https://hukamnas.com/dampak-sekularisme-terhadap-kehidupan.

    Hidup dalam Terang Sabda. “Formalisme Agama.” Accessed October 29, 2021. https://www.terang-sabda.com/2008/04/formalisme-agama.html.

    Mahmuddin, Mahmuddin. “FORMALISME AGAMA DALAM PERSFEKTIF GERAKAN SOSIAL: Prospek Dan Tantangan Di Masa Depan.” Jurnal Diskursus Islam 3, no. 1 (2015). https://doi.org/10.24252/jdi.v3i1.194.

    “Masalah Sekularisme Dan Dampaknya Dalam Hubungan Mayoritas-Minoritas – CRCS UGM.” Accessed October 29, 2021. https://crcs.ugm.ac.id/masalah-sekularisme-dan-dampaknya-dalam-hubungan-mayoritas-minoritas/.

    Character Building. “Menimbang Sekularisme Dari Sudut Pandang Agama (Sebuah Refleksi).” Accessed October 29, 2021. https://binus.ac.id/character-building/2020/05/menimbang-sekularisme-dari-sudut-pandang-agama-sebuah-refleksi/.

    “Pengertian Sekularisme - Ciri, Pendidikan, Bahaya, Para Ahli.” Accessed October 29, 2021. https://www.gurupendidikan.co.id/pengertian-sekularisme/.

    “Sekularisme, Trauma Masa Lalu Bangsa Eropa - Jurnalposmedia.Com.” Accessed October 29, 2021. https://jurnalposmedia.com/sekularisme-trauma-masa-lalu-bangsa-eropa/.

    Suraji, Robertus. “FORMALISME KEHIDUPAN BERAGAMA.” Jurnal Filsafat, 2017, 12.

		
\end{document}

\documentclass[a4paper, 11pt]{article}
\usepackage[margin=2cm]{geometry}
\usepackage{plex-otf}
\usepackage{listings}
\usepackage{xcolor}

\setlength{\parskip}{0pt} 
\setlength{\parindent}{0pt}

\newcommand{\mytitle}{Week 9 Assignments}
\newcommand{\theauthor}{Rivo Juicer Wowor}
\newcommand{\affiliation}{00000059635}

\usepackage{fancyhdr}
\fancypagestyle{plain}{%
  \fancyhf{}%
  \lhead{\footnotesize{\textbf{\mytitle}}}%
  \fancyfoot[R]{\thepage}%
  \fancyfoot[L]{\footnotesize{\theauthor \hspace{1pt} (\affiliation)}}
  \renewcommand{\headrulewidth}{0.4pt}% Line at the header invisible
  \renewcommand{\footrulewidth}{0.4pt}% Footer line not visible with 0pt
}

\lstdefinestyle{code}{
    breaklines=true,
    language=[ANSI]C,
    numbers=left,
    numbersep=12pt,
    numberstyle=\footnotesize,
    tabsize=2,
    showstringspaces=false,
	basicstyle=\normalsize \linespread{1.2},
    frame=lines,
    postbreak=\mbox{\textcolor{red}{$\hookrightarrow$}\space},
}

\lstdefinestyle{output}{
    breaklines=true,
    frame=shadowbox
}
 
\pagestyle{plain}

\begin{document}
    \section*{Practice 1}
    \begin{minipage}[t]{0.5\textwidth}
        \large \textbf{Code}
        \begin{lstlisting}[style=code]
#include<stdio.h>

int main(void){
    int x;
    x = 5 * 8 - 7 + 2 * 3 - 3 + 4 / 2 + 2;
    printf("%i\n", x);
    return 0;
}
        \end{lstlisting}
    \end{minipage}
    \hspace{0.5cm}
    \begin{minipage}[t]{0.5\textwidth}
        \large \textbf{Output}
        \begin{lstlisting}[style=output]
40
        \end{lstlisting}
    \end{minipage}

    \section*{Practice 2}
    \begin{minipage}[t]{0.5\textwidth}
        \large \textbf{Code}
        \begin{lstlisting}[style=code]
#include<stdio.h>

int main(void){
    int x = 14, y = 5;
    y = x++ % 3;
    printf("%i\n", x);
    printf("%i\n", y);
}
        \end{lstlisting}
    \end{minipage}
    \hspace{0.5cm}
    \begin{minipage}[t]{0.5\textwidth}
        \large \textbf{Output}
        \begin{lstlisting}[style=output]
15
2
        \end{lstlisting}
    \end{minipage}

    \section*{Practice 3}
    \begin{minipage}[t]{0.5\textwidth}
        \large \textbf{Code}
        \begin{lstlisting}[style=code]
#include<stdio.h>

int main(void){
    int x = 3, y = 5;
    y *= 12 / ++x;
    printf("%i\n", x);
    printf("%i\n", y);
}
        \end{lstlisting}
    \end{minipage}
    \hspace{0.5cm}
    \begin{minipage}[t]{0.5\textwidth}
        \large \textbf{Output}
        \begin{lstlisting}[style=output]
4
15
        \end{lstlisting}
    \end{minipage}

    \section*{Practice 4}
    \begin{minipage}[t]{0.5\textwidth}
        \large \textbf{Code}
        \begin{lstlisting}[style=code]
#include<stdio.h>

int main(void){
    int x = 10, y = 20, z = 30;
    x *= y += ++z;

    printf("%i\n", x);
    printf("%i\n", y);
    printf("%i\n", z);
}
        \end{lstlisting}
    \end{minipage}
    \hspace{0.5cm}
    \begin{minipage}[t]{0.5\textwidth}
        \large \textbf{Output}
        \begin{lstlisting}[style=output]
510
51
31
        \end{lstlisting}
    \end{minipage}
    \section*{Practice 5}
    \begin{minipage}[t]{0.5\textwidth}
        \large \textbf{Code}
        \begin{lstlisting}[style=code]
#include<stdio.h>

int main(void){
    int x = 10, y;
    y = x++ + ++x;

    printf("%i\n", x);
    printf("%i\n", y);
}
        \end{lstlisting}
    \end{minipage}
    \hspace{0.5cm}
    \begin{minipage}[t]{0.5\textwidth}
        \large \textbf{Output}
        \begin{lstlisting}[style=output]
12
22
        \end{lstlisting}
    \end{minipage}

    \section*{Practice 6}
    \begin{minipage}[t]{0.5\textwidth}
        \large \textbf{Code}
        \begin{lstlisting}[style=code]
#include<stdio.h>

int main(void){
    int x = 8, y;
    y = ~x;

    printf("%i\n", x);
    printf("%i\n", y);
}
        \end{lstlisting}
    \end{minipage}
    \hspace{0.5cm}
    \begin{minipage}[t]{0.5\textwidth}
        \large \textbf{Output}
        \begin{lstlisting}[style=output]
8
-9
        \end{lstlisting}
    \end{minipage}

    \section*{Practice 7}
    \begin{minipage}[t]{0.5\textwidth}
        \large \textbf{Code}
        \begin{lstlisting}[style=code]
#include<stdio.h>

int main(void){
    int x = 8, y = 6, p, q, r;
    p = x & y;
    q = x | y;
    r = x ^ y;

    printf("%i\n", p);
    printf("%i\n", q);
    printf("%i\n", r); 
}
        \end{lstlisting}
    \end{minipage}
    \hspace{0.5cm}
    \begin{minipage}[t]{0.5\textwidth}
        \large \textbf{Output}
        \begin{lstlisting}[style=output]
0
14
14
        \end{lstlisting}
    \end{minipage}

    \section*{Practice 8}
    \begin{minipage}[t]{0.5\textwidth}
        \large \textbf{Code}
        \begin{lstlisting}[style=code]
#include<stdio.h>

int main(void){
    int x = 8, y = 12, z;
    z = x << y / 4;
    y >>= x / 4;

    printf("%i\n", x);
    printf("%i\n", y);
    printf("%i\n", z);
}
        \end{lstlisting}
    \end{minipage}
    \hspace{0.5cm}
    \begin{minipage}[t]{0.5\textwidth}
        \large \textbf{Output}
        \begin{lstlisting}[style=output]
8
3
64
        \end{lstlisting}
    \end{minipage}

    \section*{Practice 9}
    \begin{minipage}[t]{0.5\textwidth}
        \large \textbf{Code}
        \begin{lstlisting}[style=code]
#include<stdio.h>

int main(void){
    int score = 48;
    score >= 55 ? printf("Pass\n") : printf("Fail\n");
}
        \end{lstlisting}
    \end{minipage}
    \hspace{0.5cm}
    \begin{minipage}[t]{0.5\textwidth}
        \large \textbf{Output}
        \begin{lstlisting}[style=output]
Fail
        \end{lstlisting}
    \end{minipage}

    \section*{Practice 10}
    \begin{minipage}[t]{0.5\textwidth}
        \large \textbf{Code}
        \begin{lstlisting}[style=code]
#include<stdio.h>

int main(void){
    int x, y, sum;
    float avg;

    printf("Number 1: ");
    scanf("%i", &x);
    printf("Number 2: ");
    scanf("%i", &y);
    printf("\n");

    sum = x + y;
    avg = (float) sum / 2;

    printf("Sum = %i\n", sum);
    printf("Average = %.2f\n", avg);
}
        \end{lstlisting}
    \end{minipage}
    \hspace{0.5cm}
    \begin{minipage}[t]{0.5\textwidth}
        \large \textbf{Output}
        \begin{lstlisting}[style=output]
$ Number 1: 5
$ Number 2: 10

  Sum = 15
  Average = 7.50
        \end{lstlisting}
    \end{minipage}

    \section*{Practice 11}
    \begin{minipage}[t]{0.5\textwidth}
        \large \textbf{Code}
        \begin{lstlisting}[style=code]
#include<stdio.h>

int main(void){
    const float pi = 3.14159;
    int radius, diameter;
    float circum, area;

    printf("Radius: ");
    scanf("%i", &radius);

    diameter = radius * 2;
    circum = pi * (float) diameter;
    area = pi * radius * radius;

    printf("Diameter = %i\n", diameter);
    printf("Circumference = %.2f\n", circum);
    printf("Area = %.2f\n", area);
}
        \end{lstlisting}
    \end{minipage}
    \hspace{0.5cm}
    \begin{minipage}[t]{0.5\textwidth}
        \large \textbf{Output}
        \begin{lstlisting}[style=output]
$ Radius: 5
  Diameter = 10
  Circumference = 31.42
  Area = 78.54
        \end{lstlisting}
    \end{minipage}

\end{document}
\documentclass[a4paper, 10pt]{article}
\usepackage[margin=2cm]{geometry}
\usepackage{plex-otf}
\usepackage{listings}
\usepackage{xcolor}

\setlength{\parskip}{0pt} 
\setlength{\parindent}{0pt}

\newcommand{\mytitle}{Week 11 Assignments}
\newcommand{\theauthor}{Rivo Juicer Wowor}
\newcommand{\affiliation}{00000059635}

\usepackage{fancyhdr}
\fancypagestyle{plain}{%
  \fancyhf{}%
  \lhead{\footnotesize{\textbf{\mytitle}}}%
  \fancyfoot[R]{\thepage}%
  \fancyfoot[L]{\footnotesize{\theauthor \hspace{1pt} (\affiliation)}}
  \renewcommand{\headrulewidth}{0.4pt}% Line at the header invisible
  \renewcommand{\footrulewidth}{0.4pt}% Footer line not visible with 0pt
}

\lstdefinestyle{code}{
    breaklines=true,
    language=[ANSI]C,
    numbers=left,
    numbersep=12pt,
    numberstyle=\footnotesize,
    tabsize=2,
    showstringspaces=false,
	basicstyle=\normalsize \linespread{1.2},
    frame=lines,
    postbreak=\mbox{\textcolor{red}{$\hookrightarrow$}\space},
    morecomment=[l][\textcolor{lightgray}]{//}
}

\lstdefinestyle{output}{
    breaklines=true,
    frame=shadowbox,
    basicstyle=\normalsize,
    morecomment=[l][\textbf]{>},
    morecomment=[l][\emph]{./}
}
 
\pagestyle{plain}

\begin{document}

    \section*{Blood example}
    \large \textbf{Code}
    \begin{lstlisting}[style=code]
#include <stdio.h>

int isdigit(char *input){
    int i = 0;
    while (input[i] != '\0') {
      // check if element i of a string is a character based on ASCII code.
      if (!(input[i] >= 48 && input[i] <= 57)) return 1; 
      i++;
    }
    return 0;
}

int convertToInt(char *str){
  int i = 0, result = 0;
  while (str[i] != '\0') {
    // Each i element in str variable is subtracted by 48 ('0' character on ASCII)
    // to get the real number and then added to result variable.
    result = result * 10 + str[i] - '0'; 
    i++;
  }
  return result;
}

int main() {
  char input[10]; // Initialize a string called 'input'
  scanf("%s", input); // Get input

  if (isdigit(input) == 1) { printf("Input contains character(s) other than numbers\n"); return 1; }

  // Convert input to integer
  int systolicBloodPressure = convertToInt(input); 

  if (systolicBloodPressure > 140) printf("Hypertension\n");
  else if(systolicBloodPressure > 120) printf("Pre-hypertension\n");
  else if(systolicBloodPressure > 90) printf("Normal\n");
  else printf("Hypotension\n");

  return 0;
}
    \end{lstlisting}

    \vspace{0.5cm}

    \large \textbf{Output}
    \begin{lstlisting}[style=output]
./blood
> 126
Pre-hypertension

./blood
> 1A5
Input contains character(s) other than numbers
    \end{lstlisting}

    \section*{Practice 1}
    \begin{minipage}[t]{0.55\textwidth}
        \large \textbf{Code}
        \begin{lstlisting}[style=code]
#include<stdio.h>

int main(void){
    int noise;

    scanf("%d", &noise);
    if(noise <= 50) printf("Quiet\n");
    else if(noise <= 70) printf("Intrusive\n");
    else if(noise <= 90) printf("Annoying\n");
    else if(noise <= 110) printf("Very Annoying\n");
    else printf("Uncomfortable\n");

    return 0;
}
        \end{lstlisting}
    \end{minipage}
    \hspace{0.5cm}
    \begin{minipage}[t]{0.4\textwidth}
        \large \textbf{Output}
        \begin{lstlisting}[style=output]
> 80
Annoying
        \end{lstlisting}
    \end{minipage} 

    \section*{Practice 2}
    \begin{minipage}[t]{0.55\textwidth}
        \large \textbf{Code}
        \begin{lstlisting}[style=code]
#include<stdio.h>

int main(void){
    int grade;

    scanf("%d", &grade);
    if(grade >= 55)
        printf("Passed\n");
    else {
        printf("Failed\n");
        printf("You must take this course again\n");
    }
    return 0;
}
        \end{lstlisting}
    \end{minipage}
    \hspace{0.5cm}
    \begin{minipage}[t]{0.4\textwidth}
        \large \textbf{Output}
        \begin{lstlisting}[style=output]
> 85
Passed
        \end{lstlisting}
    \end{minipage}

    \section*{Practice 3}
    \begin{minipage}[t]{0.55\textwidth}
        \large \textbf{Code}
        \begin{lstlisting}[style=code]
#include<stdio.h>

int main(void){
    char color;

    color = getchar();
    switch(color)
    {
        case 'R': printf("Red\n"); break;
        case 'G': printf("Green\n"); break;
        case 'B': printf("Blue\n"); break;
    }

    return 0;
}
        \end{lstlisting}
    \end{minipage}
    \hspace{0.5cm}
    \begin{minipage}[t]{0.4\textwidth}
        \large \textbf{Output}
        \begin{lstlisting}[style=output]
> R
Red
        \end{lstlisting}
    \end{minipage}

    \section*{Practice 4}
    \begin{minipage}[t]{0.55\textwidth}
        \large \textbf{Code}
        \begin{lstlisting}[style=code]
#include<stdio.h>

int main(void){
    int ph;
    printf("Kandungan PH: ");
    scanf("%i", &ph);

    if (ph > 7) {
        if (ph < 12) {
            printf("Alkaline");
        } else {
            printf("Very Alkaline");
        }
    } else {
        if (ph == 7) {
            printf("Neutral");
        } else if (ph > 2) printf("Acidic");
        else printf("Very Acidic");
    }
}
        \end{lstlisting}
    \end{minipage}
    \hspace{0.5cm}
    \begin{minipage}[t]{0.4\textwidth}
        \large \textbf{Output}
        \begin{lstlisting}[style=output]
> Kandungan PH: 18
Very Alkaline
        \end{lstlisting}
    \end{minipage}

    \section*{Practice 5}
    \begin{minipage}[t]{0.55\textwidth}
        \large \textbf{Code}
        \begin{lstlisting}[style=code]
#include<stdio.h>

int main(void){
    int lumens, watts;
    scanf("Watts: %i", &watts);
    switch(watts) {
        case 15: lumens = 125; break;
        case 25: lumens = 215; break;
        case 40: lumens = 500; break;
        case 60: lumens = 880; break;
        case 75: lumens = 1000; break;
        case 100: lumens = 1675; break;
        default: lumens = -1; break;
    }
    printf("%i", lumens);
}
        \end{lstlisting}
    \end{minipage}
    \hspace{0.5cm}
    \begin{minipage}[t]{0.4\textwidth}
        \large \textbf{Output}
        \begin{lstlisting}[style=output]
./practice5
> Watts: 25
215

./practice5
> Watts: 78
-1
        \end{lstlisting}
    \end{minipage}

    \section*{Practice 6}
    \begin{minipage}[t]{0.55\textwidth}
        \large \textbf{Code}
        \begin{lstlisting}[style=code]
#include<stdio.h>

int main(void){
    int wind;
    char* category;
    printf("Wind speed: ");
    scanf("%i", &wind);
    if (wind < 25) category = "Not a strong wind";
    else if (wind >= 25 && wind < 39) category = "Strong wind";
    else if (wind >= 39 && wind < 55) category = "Gale";
    else if (wind >= 55 && wind < 73) category = "Whole gale";
    else category = "Hurricane";

    printf("Category: %s\n", category);
}
        \end{lstlisting}
    \end{minipage}
    \hspace{0.5cm}
    \begin{minipage}[t]{0.4\textwidth}
        \large \textbf{Output}
        \begin{lstlisting}[style=output]
./practice6
> Wind speed: 75
Category: Hurricane

./practice6
> Wind speed: 30
Category: Strong wind
        \end{lstlisting}
    \end{minipage}
\end{document}
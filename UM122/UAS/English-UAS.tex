\documentclass[12pt]{article}
\usepackage[a4paper, hmargin=0.5in, vmargin=1in]{geometry}
\usepackage{microtype}
\usepackage{plex-otf}
% \usepackage{fontspec}
% \setmainfont{Times New Roman}

\usepackage[english]{babel}
\usepackage{csquotes}
\usepackage{titling}
\usepackage{enumitem}

\usepackage[backend=biber, sorting=none, style=apa, refsection=section]{biblatex}
\addbibresource{../../ref/library.bib}
\usepackage{hyperref}

\renewcommand{\baselinestretch}{1.5}
\setlength{\parskip}{1em}

\newcommand{\mytitle}{UM122 - English 1}
\newcommand{\theauthor}{Rivo Juicer Wowor}
\newcommand{\affiliation}{00000059635}

\usepackage{fancyhdr}
\fancypagestyle{plain}{%
  \fancyhf{}%
  \lhead{\footnotesize{\textbf{\mytitle}}}%
  \fancyfoot[R]{\thepage}%
  \fancyfoot[L]{\footnotesize{\theauthor \hspace{1pt} (\affiliation)}}
}
 
\pagestyle{plain}

\begin{document}
    \section*{Part 1}
    \subsection*{Questions}
    \begin{enumerate}
        \item According to the text, Nadiem Anwar Makarim explained the details
        of the PPKS Permendikbudristek, which later drew polemics from some communities.
        In your perspective, what polemics would probably emerge regarding the
        PPKS Permendikbudristek or regulation of sexual violence in university.
        
        Why do you think this regulation could create polemics? Please elaborate
        your answers with strong arguments, real examples, and valid sources or references!

        \item In your perspective, do you agree that formal sex education must
        be implemented in university curriculum? Please elaborate your answers
        with arguments, valid sources or references!
    \end{enumerate}

    \subsection*{Answers}
    \begin{refsection}
        
        \begin{enumerate}
            \item 
            % From my perspective, there's a few polemics that would be emerge regarding to this PPKS.  
            %   - Many people would think this bill will encourage consent and free sex
            %   - Few phrases that in this bill would be misinterpreted to legalizes extramarital and adultery
            
            % These polemics happened because:
            %   - People have different opinion about what sexual violence is about.
            %   - Talking about sex is mostly a taboo in Indonesia, so a bill about sexual violence would create many
            %     misunderstanding among people.
            
            From my perspective, there would be few polemics that would be
            emerge related to this bill. First, few phrases inside this bill can
            be misunderstood. If you read the bill in the Article 5 \parencite{kementerianpendidikandankebudayaanPeraturanMenteri302021},
            there will be a lot of \emph{"without the victim's consent"} phrase in the
            points. And I think that this phrase would create many problems
            because this phrase can be misinterpreted by alot of people to
            encourage consent and free sex. 
            
            These polemics could be happened because many people have different
            opinion about what sexual violence is about. Some people thought
            that sexual violence only happened when there is intercourse without
            consent \parencite{chandraNadiemMakarimSexual2021}. And anything other than that (like catcalling, sexual
            assault, etc) to them is a relatively an okay thing to do. Another reason
            is talking about sex is mostly a taboo thing in Indonesia, so a bill
            about sexual violence would create many misunderstanding among people.
            Many people would argue that this bill would legalize extramarital
            and adultery in Indonesia's universities. And I could think these
            would happened because their opinion would be very biased by
            their beliefs.
            
            \item
            Yes, I think sex education should be implemented in formal university 
            curriculum. In \emph{International Technical Guidance of Sexuality Education}
            guidelines by WHO, Formal sex education is important because it
            enables young people to develop accurate knowledge, positive values,
            respect for human rights, gender equality, and diversity that
            contributes to safe, healthy, positive relationship \parencite{whoInternationalTechnicalGuidance2018}.
            Sex education is not only need to be implemented in universities
            curriculum, but also in elementary to high school curriculum too. A
            research by \citeauthor[]{santelliDoesSexEducation2018} in
            \citeyear{santelliDoesSexEducation2018} shows that the lack of
            institutional education led students to have alternative solutions to
            learn more about sex and sexuality, and a majority of sexual abuse
            and violence in college happened because sex education before
            college was awkward and poorly delivered. That's why sex education
            need to be implemented on every curriculum for students in order to
            avoid sex assault and violence cases in Indonesia's education institution.
        \end{enumerate}
        \nocite{
            bbcnewsindonesiaMengapaTanpaPersetujuan2021,
        }
        \pagebreak
        \printbibliography[title=First Part References]
    \end{refsection}

    \pagebreak
    \section*{Part 2}
    \subsection*{Outline}
    \begin{enumerate}[label*=\arabic*.]
        \item {\large\textbf{Introductory Paragraph}}
        \begin{enumerate}[label*=\arabic*.]
            \item \textbf{Background}
            \begin{enumerate}
                \item 68\% of Indonesian women did not feel pretty when they did not use cosmetics
                \item Indonesian people's stigma about beauty standard is unrealistic and too high
                \item Many of Indonesian women become less confident about their own beauty.
            \end{enumerate}
            \item \textbf{Thesis Statement} \\
        Indonesian women shouldn't follow any beauty standards that have been set by society
        \end{enumerate}
    \item {\large\textbf{Body Paragraph}}
        \begin{enumerate}[label*=\arabic*.]
        \item \textbf{Don't attached to what media shows you, but be yourself instead}
            \begin{enumerate}[label*=\arabic*.]
            \item Media portrayal of beauty who changed people perception
                \begin{enumerate}[label*=\arabic*.]
                \item Indonesia's Beauty product commercial potray woman with lighter skin color is prettier than woman with darker skin color
                \item Not only in Indonesia, many women reported that they tend to compare their own appearance negatively with their peer group and with celebrities in social media.
                \end{enumerate}
            \item Indonesian women has unique natural beauty (cite from Wirasari)
            \item Beauty of women should also be defines in various way (cite from Prianti)
            \item Therefore, women of Indonesia should be more confident about their own beauty
            \end{enumerate}
        \item \textbf{Inner beauty is a lot more important than outer beauty}
            \begin{enumerate}[label*=\arabic*.]
            \item Inner beauty = kindness, diligent, and/or achievements
                \begin{enumerate}[label*=\arabic*.]
                \item Add also inner beauty research by Lisa Katharin Schmalzried
                \end{enumerate}
            \item Why inner beauty?
                \begin{enumerate}[label*=\arabic*.]
                \item Not everyone can have inner beauty
                \item Have longer lasting and eternal effect
                \item Most people appreciate inner beauty more
                \end{enumerate}
            \item Few steps to achieve inner beauty (cite SehatQ)
                \begin{enumerate}[label*=\arabic*.]
                \item Thinking more positively about your personality instead of your looks.
                \item Know yourself, focus more on your life quality.
                \end{enumerate}
            \end{enumerate}
        \end{enumerate}
    \item {\large\textbf{Conclusion}} \\
            There's still a lot of women in Indonesia that are not confident by
            their own body because of society's stigma about beauty. Stigma like
            having slim body and bright skin to be "beautiful" is what concerning
            to a lot of women in Indonesia and therefore, creating a bad habit
            and bad experience for them to chase that unrealistic beauty standards.
            One of the reason on why all of this happened because how media portray
            beauty in television shows, ads, and social media. Media often linked
            beautifulness with foreign women beauty characteristics rather than
            using Indonesian own unique natural beauty. That's why, Indonesian
            women should be confident about themselves. Also, more women should
            focused more on inner beauty rather than outer beauty because on how
            long lasting the effect it has. Therefore, Indonesian women should
            not be bother anymore by any of beauty standards set by society.
    \end{enumerate}

    \title{Why Indonesian Women should not mind about Unrealistic Beauty Standards}
    \author{Rivo Juicer Wowor (00000059635)}
    \date{}
    \maketitle
    \begin{refsection}
        In this digital age, there's a lot of things that changed and shaped our
        society by media. And beauty is one of those things that has changed. But
        because of that, our society set the standards of Indonesian women's beauty
        to the point where it is unrealistic and not representing women as an
        Indonesian at all. Few stigma that grows inside our society mind are like
        how light your skin is, how slim your body is, and how \emph{young} your
        face is like. And those stigma create a huge impact on how Indonesian women
        saw themselves. One of the impact of this fact is 68\% of Indonesian women
    did not feel pretty when they didn't use cosmetics based on Beauty Confidence
    Report survey in 2018 \parencite{maleSurvei84Persen}.
    Therefore, I propose that Indonesian women should not mind or follow any
    unrealistic beauty standards that have been set by society and media.

    First, Indonesian women should not attached or focused too much on what social
    media, or even media in general shows them about beauty, but instead be themselves
    as it is. Because media portrayal is one of the main culprit on how people 
    perceive women's beauty. We can see it on Indonesia's beauty product
    commercial who make a stigma where a women with lighter skin is prettier than
    the women with a darker skin color \parencite{prianti2013indonesian}. There's also social media impact on
    beauty standard; not only in Indonesia, but in the whole world. Based on 
    \citeauthor{fardoulyNegativeComparisonsOne2015} survey in \citeyear{fardoulyNegativeComparisonsOne2015},
    many women reported that they often tend to compare their own appearance
    negatively with their friends and with celebrities in social media. 
    Indonesia is a diverse country filled with variety of people with different
    color, race, and characteristics; and this includes Indonesian women who have
    their unique natural beauty compared to other country \parencite{wirasariKAJIANKECANTIKANKAUM2016}.
    And not only that, \citeauthor{prianti2013indonesian} in her paper said that 
    beauty of women should also be defined in various way, not by a set list
    of fixed characteristics.
    Then, how do you gain confidence? In her interview with
    \citeauthor{kumparanwomanBagaimanaCaraMeningkatkan2019}, Pingkan C.B. Rumondor;
    a clinical psychologist said there are three ways a woman can do to gain self-confidence.
    First, you need to set goals in order to gain successfull experience. This 
    experience will contribute to increasing your self-confidence. Then, you'll 
    need a positive feedback from people. Those feedback can be retrieved by asking
    your close relatives about your personality and your quality in their perspectives.
    And those feedback will indirectly helps you to achieve your goals and gain
    self confidence. And lastly, find a mentor or a role model who are successful
    in their career. Because a mentor will really helps on guiding you to be 
    successful just like them.
    
    Besides confidence, I think that women's inner beauty is a lot more important
    than outer beauty. But before that, what is an \emph{"inner beauty"}? According
    to Plato, inner beauty is a beauty based on how their action is morally good
    \parencite{wisnubrataCaraMemancarkanInner2020}. Another definition by \citeauthor{schmalzriedInnerBeautyFriendshipHypothesis2013}.
    who defines inner beauty as
    \begin{quote}
        \emph{a person who insofar inwardly beautiful
        because she effortlessly or out of love and affection does what is morally
        required or praiseworthy.}
    \end{quote}
    And by those two definition, it concludes that inner beauty is a beauty where 
    kindness, moral, and achievements are more taken into account, rather than 
    how beautiful and sexy you are.
    So, why inner beauty is a lot more important than outer beauty? According to
    an article by \citeauthor[]{fimela.comIniAlasanMengapa2018}, there are a few
    reasons. First, Not everyone can have inner beauty. It means that a lot of people 
    only have a good-looking face, but not a good heart or a big confidence. 
    So, a person who do have inner beauty can looks different than the people
    who don't. Another one is that inner beauty has a very long-lasting effect;
    even eternals, compared to outer beauty. This is because inner beauty is
    measured by their personality and their heart, and these two elements are 
    very hard to change overtime. Compared that with a good-looking face. As we're
    getting older, our face will get wrinkles and other things that we can't avoid.
    And the last point is, inner beauty is more appreciated by people than outer
    beauty. For example, when applying for jobs, people are not looking by your
    face to hire you, but by how good your personality is. Because looks can
    be polished by make-up, but not personality.
    There are few steps that Indonesian women could do to achieve inner beauty,
    based on article written by SehatQ. First is you should thinking more positively 
    about your personality rather to think about how do you look. Because it's easier 
    to think about personality and it has a lot of benefits for your health too.
    Another one is knowing yourself. Knowing in here means that women should 
    focus more on their life quality. And by knowing themself, they will find 
    meanings about happiness and satisfaction in their life.
    And lastly, just like my first point is to be yourself. Don't think too much
    about other people's look and comparing to yourself. Don't be jealous because 
    they have been gifted a really pretty face compared to your regular face. But 
    instead, be yourself. Be what you are. And focus on things that really matters.
    Things like your hobby, your passion, and your future. Rich yourself with 
    new experience and create new memories for your life. Because by doing that,
    you will find your inner beauty eventually.
    
    There's still a lot of women in Indonesia that are not confident by their
    own body because of society's stigma about beauty. Stigma like having slim
    body and bright skin to be "beautiful" is what concerning to a lot of women 
    in Indonesia and therefore, creating a bad habit and bad experience for them
    to chase that unrealistic beauty standards. One of the reason on why all of
    this happened because how media portray beauty in television shows, ads, and
    social media. Media often linked beautifulness with foreign women beauty
    characteristics rather than using Indonesian own unique natural beauty.
    That's why, Indonesian women should be confident about themselves.
    Also, more women should focused more on inner beauty rather than outer
    beauty. Because inner beauty has a very long lasting effect rather than outer
    beauty, and most people appreciate inner beauty more. That's why kind people 
    is often hard to find these days rather than good-looking people. 
    Therefore, Indonesian women should not be bother anymore by any of beauty
    standards set by society.
    
    \pagebreak
    \nocite{
        fardoulyNegativeComparisonsOne2015,
        fimela.comIniAlasanMengapa2018,
        InnerBeautyLebih,
        islameyWacanaStandarKecantikan2020,
        kumparanwomanBagaimanaCaraMeningkatkan2019,
        maleSurvei84Persen,
        mediaTakHarusKulit2021,
        oakesComplicatedTruthSocial,
        obioraDarkSideSocial0100,
        prianti2013indonesian,
        puspitasariDiscourseShiftingLocal2020,
        rachmanBalancingOuterInner2013,
        ratnaningtiasCantikDiInstagram2018,
        rizkiyahSTRATEGICOPINGPEREMPUAN2020,
        schmalzriedInnerBeautyFriendshipHypothesis2013,
        wirasariKAJIANKECANTIKANKAUM2016,
        wisnubrataCaraMemancarkanInner2020}
        \printbibliography[title=Essay References]
    \end{refsection}

    


\end{document}